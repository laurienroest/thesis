\documentclass[12pt,a4paper]{article}
%\documentclass[11pt]{iopart}

\usepackage[colorlinks=true, linkcolor=black!50!blue, urlcolor=blue, citecolor=blue, anchorcolor=blue]{hyperref}
\usepackage[font=small,labelfont=bf,margin=0mm,labelsep=period,tableposition=top]{caption}

\usepackage[a4paper,top=2cm,bottom=2cm,left=2.5cm,right=2.5cm,bindingoffset=0mm]{geometry}

\usepackage{graphicx}
\usepackage{float}
\usepackage{afterpage}
\usepackage{epsfig,cite}
\usepackage{amssymb}
\usepackage{amsmath}
\usepackage{bm}
\usepackage{dsfont}
\usepackage{multirow}
\usepackage{url}
\usepackage{xcolor}
\usepackage{float}
\usepackage{afterpage}
\usepackage{ulem}
\usepackage{afterpage}


\usepackage{url}
\usepackage{hyperref}

\usepackage{multirow,booktabs,multirow}

%\bibliographystyle{iopart-num}
\bibliographystyle{JHEP}

%%%%%%%%%%%%%%%%%%%%%%%%%%%%%%%%%%%%%%%%%%%%%%%%%%%%%%%%%%%%%

\def\smallfrac#1#2{\hbox{$\frac{#1}{#2}$}}
\newcommand{\be}{\begin{equation}}
\newcommand{\ee}{\end{equation}}
\newcommand{\bea}{\begin{eqnarray}}
\newcommand{\eea}{\end{eqnarray}}
\newcommand{\ei}{\end{itemize}}
\newcommand{\ben}{\begin{enumerate}}
\newcommand{\een}{\end{enumerate}}
\newcommand{\la}{\left\langle}
\newcommand{\ra}{\right\rangle}
\newcommand{\lc}{\left[}
  \newcommand{\tr}{\toprule}
  \newcommand{\mr}{\midrule}
  \newcommand{\br}{\bottomrule}
\newcommand{\rc}{\right]}
\newcommand{\lp}{\left(}
\newcommand{\rp}{\right)}
\newcommand{\as}{\alpha_s}
\newcommand{\aq}{\alpha_s\left( Q^2 \right)}
\newcommand{\amz}{\alpha_s\left( M_Z^2 \right)}
\newcommand{\aqq}{\alpha_s \left( Q^2_0 \right)}
\newcommand{\aqz}{\alpha_s \left( Q^2_0 \right)}
\def\toinf#1{\mathrel{\mathop{\sim}\limits_{\scriptscriptstyle
{#1\rightarrow\infty }}}}
\def\tozero#1{\mathrel{\mathop{\sim}\limits_{\scriptscriptstyle
{#1\rightarrow0 }}}}
\def\toone#1{\mathrel{\mathop{\sim}\limits_{\scriptscriptstyle
{#1\rightarrow1 }}}}
\def\frac#1#2{{{#1}\over {#2}}}
\def\gsim{\mathrel{\rlap{\lower4pt\hbox{\hskip1pt$\sim$}}
    \raise1pt\hbox{$>$}}}       
\def\lsim{\mathrel{\rlap{\lower4pt\hbox{\hskip1pt$\sim$}}
    \raise1pt\hbox{$<$}}}       
\newcommand{\mrexp}{\mathrm{exp}}
\newcommand{\dat}{\mathrm{dat}}
\newcommand{\one}{\mathrm{(1)}}
\newcommand{\two}{\mathrm{(2)}}
\newcommand{\art}{\mathrm{art}}
\newcommand{\rep}{\mathrm{rep}}
\newcommand{\net}{\mathrm{net}}
\newcommand{\stopp}{\mathrm{stop}}
\newcommand{\sys}{\mathrm{sys}}
\newcommand{\stat}{\mathrm{stat}}
\newcommand{\diag}{\mathrm{diag}}
\newcommand{\pdf}{\mathrm{pdf}}
\newcommand{\tot}{\mathrm{tot}}
\newcommand{\minn}{\mathrm{min}}
\newcommand{\mut}{\mathrm{mut}}
\newcommand{\partt}{\mathrm{part}}
\newcommand{\dof}{\mathrm{dof}}
\newcommand{\NS}{\mathrm{NS}}
\newcommand{\cov}{\mathrm{cov}}
\newcommand{\gen}{\mathrm{gen}}
\newcommand{\cut}{\mathrm{cut}}
\newcommand{\parr}{\mathrm{par}}
\newcommand{\val}{\mathrm{val}}
\newcommand{\reff}{\mathrm{ref}}
\newcommand{\Mll}{M_{ll}}
\newcommand{\extra}{\mathrm{extra}}
\newcommand{\draft}[1]{}
\newcommand\blankpage{
    \null
    \thispagestyle{empty}
    \addtocounter{page}{-1}
    \newpage
    }
% Added by MU 
\def \a{\alpha}
\def \b{\beta}
\def \g{\gamma}
\def \z{\zeta}
\def \t{{\bf T}} % vector of theoretical predictions
\def \c{{\bf c}} % vector of coefficients of theoretical predictions
\def \y{{\bf y}} % vector of experimental data
\def \s{{\bf \sigma}} % experimental covariance matrix
% Added by JR
\def\lapprox{\lower .7ex\hbox{$\;\stackrel{\textstyle <}{\sim}\;$}}
\def\gapprox{\lower .7ex\hbox{$\;\stackrel{\textstyle >}{\sim}\;$}}
\def\half{\smallfrac{1}{2}}
\def\GeV{{\rm GeV}}
\def\TeV{{\rm TeV}}
\def\ap{{a'}}
\def\vp{{v'}}
\def\e{\epsilon}
\def\d{{\rm d}}
\def\calN{{\cal N}}
\def\shat{\hat{s}}
\def\barq{\bar{q}}
\def\qq{q \bar q}
\def\uu{u \bar u}
\def\dd{d \bar d}
\def\pp{p \bar p}
\def\xa{x_{1}}
\def\xb{x_{2}}
\def\xaa{x_{1}^{0}}
\def\xbb{x_{2}^{0}}
\def\smx{\stackrel{x\to 0}{\longrightarrow}}
\def\Li{{\rm Li}}
\numberwithin{equation}{section}
\numberwithin{figure}{section}
\numberwithin{table}{section}
\newcommand{\tmop}[1]{\ensuremath{\operatorname{#1}}}
\newcommand{\tmtextit}[1]{{\itshape{#1}}}
\newcommand{\tmtextrm}[1]{{\rmfamily{#1}}}
\newcommand{\tmtexttt}[1]{{\ttfamily{#1}}}
\usepackage{tabularx}
\newcolumntype{C}[1]{>{\centering\arraybackslash}p{#1}}
\begin{document}
\newgeometry{top=5cm,bottom=2.5cm,left=2.5cm,right=2.5cm,bindingoffset=0mm}



\vspace{5cm}

\begin{center}
  {\Huge \bf Charting the low-loss region in \\[0.3cm] Electron Energy Loss Spectroscopy \\[0.4cm] with machine learning}
\vspace{1.4cm}

{\LARGE Laurien Roest}

\vspace{1.0cm}
 


\vspace{5cm}
\end{center}


%%%%%%%%%%%%%%%%%%%%%%%%%%%%%%%%%%%%%%%%%%%%%%%%%%%%%%%%%%%%%%%%%%%%%%%
\begin{figure}[h]
\centering
\includegraphics[width=0.3\linewidth]{plots/TUDelft_logo.png} 
\end{figure}

\begin{figure}[h]
\centering
\includegraphics[width=0.3\linewidth]{plots/Nikhef_logo.png}
\end{figure}
%%%%%%%%%%%%%%%%%%%%%%%%%%%%%%%%%%%%%%%%%%%%%%%%%%%%%%%%%%%%%%%%%%%%%%%%%%

\thispagestyle{empty}
\afterpage{\blankpage}

\newpage
\newgeometry{top=2.5cm,bottom=2.5cm,left=2.5cm,right=2.5cm,bindingoffset=0mm}
\vspace{3cm}


\begin{center}
  {\Large \bf Charting the low-loss region in Electron Energy \\[0.3cm] Loss Spectroscopy with machine learning}
\vspace{1.4cm}

{\large \bf Laurien Roest \\[0.3cm]
4287320
}

\vspace{1.0cm}
 
{ to obtain the degree of Master of Science in Applied Physics \\[0.2cm]
at Delft University of Technology, \\[0.2cm]
to be defended publicly on Thursday October 1st, 2020.}

\vspace{5cm}

{\it \small Kavli Institute of Nanoscience, Delft University of Technology, 2628CJ Delft, The
  Netherlands\\[0.1cm]
Nikhef Theory Group, Science Park 105, 1098 XG Amsterdam, The
  Netherlands\\[0.1cm] }
  
\end{center}
\vspace{4cm}
Thesis committee: \\[0.1cm]
\indent Dr. S. Conesa-Boj (Supervisor, QN) \\[0.1cm]
\indent Dr. E. Greplova (QN) \\[0.1cm]
\indent Dr. ir. J.P. Hoogenboom (ImPhys)
\thispagestyle{empty}

\newpage

\noindent {\bf \large Abstract}\\[1.0cm]
Electron energy loss spectroscopy (EELS) 
provides valuable information on the structural, chemical, 
and electronic properties of materials at the nanoscale.
%
Exploiting the information contained in EEL spectra requires
reliable access to the low-loss region ($\Delta E\lsim $ few eV),
where the contribution from the zero-loss peak (ZLP) overwhelms
those from the inelastic interactions between the incident
electrons and the specimen.
%
An accurate removal of the ZLP contribution is crucial 
in order to efficiently chart and identify these low-loss features.

In this work, we deploy machine learning techniques, inspired by particle physics applications,
to realise a model-independent multidimensional determination of the ZLP
with a faithful uncertainty estimate,
which can be used to subtract its contribution to EEL spectra while
propagating the associated uncertainties.
%
Our method is then used to disentangle the 
ZLP from the sample contributions in low-loss EEL spectra acquired
in WS$_2$ nanostructures of different thicknesses.
%
This makes it possible to determine not only the value and type of the WS$_2$ bandgap
as a function of the underlying crystalline morphology of the nanostructures,
but also to characterise exciton features arising in the ultra-low-loss region.

\vspace{0.6cm}
\noindent{\it Keywords:} {\small Transmission Electron Microscopy,
Electron Energy Loss Spectroscopy, Neural Networks, Bandgap, Transition
Metal Dichalcogenides.}\\


%% The words table and figure should be written in full and not abbreviaged to tab. and fig. Do not include ‘eq.’, ‘equation’ etc before an equation number or ‘ref.’ ‘reference’ etc before a reference number.

% Table of contents
\clearpage
\tableofcontents
\newpage

% General introduction
\section{Introduction}
\label{sec:introduction}

The appreciation for the family of two-dimensional (2D) materials has grown rapidly since the 
first isolation of graphene~\cite{Novoselov:2004}.
%
The properties of these two-dimensional materials are usually very different from their 3D counterparts, 
offering functionalities for novel nanodevices that are not accessible from other heterostructures.
%
The combination of optically active semiconducting layers for light-emitting devices,
the implementation of indirect-to-direct bandgap transition 
materials for flexible electronics, the effect of chemical compositions on many-body instabilities such as
superconductivity~\cite{Novoselov:2016},
%
these are just a few examples of the relevance to exploit the properties of these 2D materials. 
%
However, realizing the full potential of any materials system requires knowing the precise 
electronic, structural and chemical information
at high resolution.

Thanks to recent instrumentation breakthroughs in electron microscopy,
such as electron monochromators~\cite{Terauchi:2005, Freitag:2005} and aberration correctors~\cite{Haider:1998},
it becomes possible to map these properties with unexampled spatial and spectral resolution.
%
Specifically by means of electron energy loss spectroscopy (EELS), it becomes possible to study the local
electronic properties of nanomaterials
down to the atomic scale~\cite{Geiger:1967}, and this way to explore various
important phenomena from the characterisation of bulk and surface plasmons~\cite{Daniels:2003, Schaffer:2008}, 
excitons~\cite{Erni:2005},
phonons~\cite{Ibach:1980}, and inter- and intra-band transitions~\cite{Rafferty:1998},
to the determination of the bandgap energy and the band structure~\cite{Stoger:2008}.

Particulary important information about the material's characteristics can be extracted from studying the
low-loss region of EEL spectra,
defined by electrons that have lost
less than a few eV ($\Delta E\lsim 5$ eV) following their inelastic interactions
with the sample.
%
However, an omnipresent feature called the zero-loss peak (ZLP) dominates
the low-loss region of the spectra. 
%
This narrow, high intensity peak is centered at energy losses
of $\Delta E\simeq 0$, is often asymmetric and its tails extend significantly 
beyond the FWHM. 
%
This peak results from the fact that 
the majority of the incident electron beam will traverse the electron-penetrable sample
either without interactions or scattering only elastically with the 
sample's crystalline lattice, therefore losing little to no energy
and ending up in the ZLP.
%
Since in the low-loss region, the contribution from the ZLP tail
overwhelms those from the inelastic interactions between
the incident electrons and the specimen,
relevant signals of low-loss phenomena such as excitons,
phonons, and intra-band transitions risk being drowned
in the ZLP tail~\cite{Abajo:2010}.
%
An accurate removal of the ZLP
contribution is thus crucial in order to efficiently chart and identify the features
of the low-loss region in EEL spectra. 


Several approaches to ZLP subtraction have been put forward in 
literature~\cite{Rafferty:2000, Erni:2005, Stoger:2008, Egerton:1996,Dorneich:1998, Benthem:2001, Lazar:2003}.
%
These are often based on specific model assumptions about the ZLP properties, specifically
concerning its parametric functional dependence on the electron energy loss $\Delta E$,
from Lorentzian~\cite{Dorneich:1998}
and power laws~\cite{Erni:2005} to more general multiple-parameter functions~\cite{Benthem:2001}.
%
Another approach is based on mirroring the $\Delta E <0$ region of the spectra, assuming
that the $\Delta E>0$ region is fully symmetric~\cite{Lazar:2003}.
%
More recent studies use integrated software applications for background subtraction 
methods~\cite{Egerton:10.1016/S0304-3991(01)00155-3, Held:2020, Granerod:2018, Fung:2020}.
%
These subtraction methods are however affected by three main limitations.
%
Firstly, they rely on specific model assumptions {\it e.g.} with
the choice of a specific fit function, introducing a methodological
bias whose size is difficult to quantify.
%
Secondly, they lack an estimate of the associated uncertainties, which in turn affects
the reliability of any physical interpretations of the low-loss region such as
band gap extraction.
%
Thirdly, methodological choices such as fitting ranges introduce a significant degree of
arbitrariness to the procedure.\\

In this work we bypass these limitations by developing a model-independent strategy
that allows for a multidimensional determination of the ZLP
with a faithful uncertainty estimate.
%
Our approach is based on machine learning (ML) techniques
developed in high-energy physics to study
quark and gluon substructure of protons
particle collisions~\cite{Ball:2008by,Ball:2012cx,Ball:2014uwa,Ball:2017nwa}.
%
This technique is based on the Monte Carlo replica method to construct a probability
distribution in the space of the experimental data (here the ZLP) and to use artificial
neural networks as unbiased interpolators to parametrise the ZLP.
%
The result is
a prediction of the ZLP intensity based on its input variables,
without the need to make specific model assumptions or approximations,
which can be used to subtract its contribution to EEL spectra while
propagating the associated uncertainties.
%
Furthermore, we can extrapolate this ZLP parametrisation to other TEM
operating conditions beyond those included in the training dataset.\\

Our work is divided into two main parts.
%
In the first, we construct a ML model of ZLP spectra taken
in vacuum that is able to accommodate an arbitrary number of input
variables corresponding to different operation settings of the TEM, 
{\it e.g.} exposure time and beam energy
%
We demonstrate how the model describes successfully the
input spectra and we assess its extrapolation for other operating
conditions that were not used for training.

In the second part, we construct a one-dimensional model
of the ZLP as a function of the energy loss from spectra acquired on
tungsten disulfide (WS$_2$) nanostructures~\cite{SabryaWS2}.
%
The resulting subtracted spectra are used to determine
the value and type of the WS$_2$ bandgap
and its dependence on the underlying crystalline morphology, 
and to demonstrate how one can exploit the ZLP-subtracted results
to characterise features arising in the very-low-loss region.

This work is organized as follows.
%
First of all, in Sects.~\ref{sec:tmd} and~\ref{sec:eels}
we discuss the intriguing class of transition metal 
dichalcogenide (TMD) materials, with emphasis on WS$_2$,
and we review the main features of the EELS technique.
%
Sect.~\ref{sec:nn} contains
the fundamentals of neural networks and the principles
of supervised machine learning.
%
In Sect.~\ref{sec:methodology} we describe our machine learning methodology
for the ZLP parametrisation.
%
Sects.~\ref{sec:results_vacuum} and~\ref{sec:results_sample} contain
our results for the ZLP parametrisation for spectra acquired
in vacuum and in samples respectively, which in the latter
case allows us to study the local band structure properties
of the WS$_2$ nanoflowers.
%
Finally in Sect.~\ref{sec:summary} we summarise
and outline possible future developments.
%
Our results have been obtained with an open-source {\sc Python} code,
{\tt EELfitter}, presented in App.~\ref{sec:installation}
together with some installation and usage instructions.

\newpage

%EELS
\section{Fundamentals of EELS}
\label{sec:eels}

Electron energy loss spectroscopy (EELS) is a method used in combination with 
transmission electron microscopy (TEM), which analyses the 
energy distribution of initially monoenergetic electrons after
they have interacted with a specimen~\cite{Egerton:1996}. 
%
In this chapter, we present an introduction to the basics
of EELS, followed by an overview of energy loss spectra and 
the limitations of the technique.\\

Electron energy loss spectroscopy was developed by James Hillier and R.F. Baker in 
the mid-1940s~\cite{Hillier:1944},
but it was only becoming more widespread in research in the 1990s 
due to improvements in microscope instrumentation. 
%
Since modern instrumentation became widely available in laboratories in the
mid-1990s, the scientific developments regarding electron microscopes grew rapidly.
%
Especially since the introduction of modern aberration correctors
and monochromated electron sources, energy resolutions of 100 meV
or even higher could be achieved~\cite{Rose:2008},
which enabled measurements of single (columns of) atoms. 
%
Transmission electron microscopy can provide structural information 
with excellent spatial resolution down to atomic dimensions. 
These structural data can be supplemented by chemical information 
from the same specimen region, obtained using analytical techniques
such as EELS.
%
EELS instrumentation is typically incorporated into a scanning
transmission electron microscope (STEM) or in a conventional TEM (c-TEM).
%
These microscope types use high energy electrons, typically 60 - 300 keV, 
to interrogate the sample. 
%
The transmitted electrons are deflected through a uniform magnetic field 
of the order of 0.01 T, generated by an electromagnet with carefully shaped polepieces. 
%
Electrons that scattered inelastically will stray from the central trajectory, 
giving rise to a greater or lesser deflection angle, 
and are sorted and detected according to their energies. 
%
The existence of different kinetic energies thus results in a fringing 
field at the EELS detector.
%
A schematic illustration of a typical EELS setup is shown in the left panel of Fig.~\ref{fig:EELS}.
%

%%%%%%%%%%%%%%%%%%%%%%%%%%%%%%%%%%%%%%%%%%%%%%%
\begin{figure}[H]
    \centering
    \includegraphics[width=0.97\textwidth]{plots/EELS.pdf}
    \caption{Left: in STEM-EELS, a magnetic
      prism is used to deflect the electron beam after crossing the sample
      so that the distribution of their energy losses $\Delta E$ can be recorded.
      %
      Right: a representative spectrum for $\Delta E \le 35$ eV acquired 
      on a WS$_2$ nanoflower~\cite{SabryaWS2} with
      the inset displaying the corresponding ZLP.
      }
    \label{fig:EELS}
\end{figure}
%%%%%%%%%%%%%%%%%%%%%%%%%%%%%%%%%%%%%%%%%%%%%%%%5

Electron energy loss spectroscopy  provides detailed information about the 
chemical components, bonding, and structure of materials down to the atomic scale.
%
Thanks to recent progress in TEM instrumentation and data acquisition, state-of-the-art 
EELS analyses benefit from
a highly competitive energy (spectral) resolution combined with an unparalleled spatial resolution.
%
Especially scanning transmission electron microscopy (STEM) equipped with a monochromator 
is extremely useful for high resolution imaging.
%%
The right panel of Fig.~\ref{fig:EELS} displays
a representative EELS spectrum in the region $\Delta E \le 35$ eV, recorded
in one of the WS$_2$ nanostructures presented in~\cite{SabryaWS2}
and which will be further discussed later onwards.

\subsection{EEL spectra}
If we are to understand how the features in electron energy loss spectra are produced, 
we must consider how the interaction of the incident electron with the sample 
contributes to the spectrum. 
%
Roughly speaking, EEL spectra can be divided into three main regions.\\

{\bf Zero-loss region.} The first is the zero-loss region, which is centered around $\Delta E=0$
and contains the contributions from electrons that are transmitted without suffering
measurable energy loss.
%
Provided the thickness of the sample is small, the greatest part of the 
incident electron beam transfers through the sample elastically, 
implicating the energy exchange is less than the experimental energy resolution. 
%
A strong and narrow intensity peak around 0 eV loss can be observed called the zero-loss peak (ZLP) or elastic peak. 
The width of the ZLP, typically 0.2-2 eV, reflects the energy distribution of the electron source.

The inset in Fig~\ref{fig:EELS} displays the ZLP, illustrating how nearby $\Delta E\simeq 0$
its magnitude is larger than the contribution from the inelastic interactions
with the sample by several orders of magnitude.
%
This can be explained by the fact that a nucleus is thousands of times more massive than an electron, 
and therefore the energy transfer involved in elastic scattering is usually negligible. 
%
The probability of elastic scattering for a single incident electron 
(per unit solid angle $\Omega$) can to a first approximation be described by 
the differential cross section~\cite{Egerton:1996},
\begin{equation}
    d\sigma_e / d\Omega = \frac{4Z^2/k_0^2T}{(\theta^2 + \theta_0^2)^2}
\end{equation}

where $Z$ is the atomic number, $k_0$ the electron wavenumber, 
and $T$ is not the temperature but the incident electron energy. 
$\theta$ is the scattering angle of the electron of interest, 
and $\theta_0$ represents the angular width of the scattered beam. \\

{\bf Low-loss region.} The second region is the low-loss region or valence region, defined for energy losses
$\Delta E \lsim 50$ eV, which contains information
about several important features such as plasmons, excitons, phonons, and
inter- and intra-band transitions.
%
Of particular relevance in this context is the ultra-low-loss region, characterised by $\Delta E \simeq$ few eV.
Here, the contributions of the ZLP and those from the inelastic interactions
with the sample are of the same order of magnitude.
%
Ranging between electron energy losses of 1 and 5 eV, 
difficultly distinctive features are the presence of 
relatively narrow exciton peaks, located close to the ZLP. 
%
The intensity of these valence features is over two orders of magnitude lower 
than the ZLP~\cite{Abajo:2010}. These peaks can be interpreted as arising from
indirect and direct exciton transitions, the latter of which allows for
bandgap determination~\cite{Stoger:2008}.

The  bandgap  refers  to  the  energy  difference between the top of the valence band 
and the bottom of the conduction band and the corresponding peak is expected to appear 
at energy losses where the joint density of states (JDOS) exhibits maxima. 
%
It has been shown by Bruley and Brown~\cite{Bruley:1987} that for parabolic bands, 
the JDOS probed by the electrons can be described by
\begin{equation}
\label{eq:bandgap}
    \rho(E) = \frac{V}{4\pi^2} \left( \frac{2m^*}{\hbar} \right)^{(3/2)} \alpha \sqrt{E-E_{bg}}
\end{equation}
for a direct band gap, where $m^*$ is the mass of the electrons and holes in the 
valence and conduction band, $E_{bg}$ is the band gap energy and $\alpha$ is the 
convergence angle of the electron beam.
%%
The nature (direct of indirect) and the band gap energy can be deduced 
from the first few eV of the energy-loss function. 
%
From a fit of the band gap onset to Eq.~\ref{eq:bandgap}, 
the value of (1/2) for a direct bandgap switches to (3/2) for an indirect bandgap,
as demonstrated in Fig.~\ref{fig:bandgap}.

\begin{figure}[H]
    \centering
    \includegraphics[width=0.8\textwidth]{plots/bandgap.png}
    \caption{Schematic diagrams showing the contributions from the JDOS and matrix elements for (a) direct and (b) indirect transitions. From a fit of the bandgap onset to Eq.~\ref{eq:bandgap}, the power of (1/2) for a direct bandgap switches
    to (3/2) for an indirect transition.
    %
    Retrieved directly from~\cite{Rafferty:1998}.}
    \label{fig:bandgap}
\end{figure}

At even lower energy losses, $\Delta E \lsim 100$ meV, vibrational modes can be revealed. 
These are the result of transmitted electrons that generate (and absorb) phonons 
while passing through the crystal. 
%
Phonon energies are of the order $k_bT$ and corresponding energy losses are below $0.1$ eV, 
which requires very high resolution spectroscopy to record it. 
%
Limited vibrational spectroscopy becomes possible in an electron microscope at around 30 meV 
energy resolution. 
%
Vibrational spectroscopy, a field that didn't exist five years ago, includes vibrational mapping, 
analyzing the momentum dependence of vibrational states and determining the local
temperature from the ratio of energy gains to losses~\cite{Krivanek:2009}.\\

{\bf Core-loss region.} The regime for $\Delta E \gsim 50$ eV is the core-loss region,
which provides compositional information
on the elements that constitute the sample. 
%
In this regime, the spectrum shows characteristic features called ionisation edges, 
formed when an inner-shell electron absorbs enough energy from a beam electron 
to be excited to a state above the Fermi level. 
%
The thicker the sample, the more prominent the ionisation edges are 
since multiple scattering events are more common. 
%
In this work however, the focus will be on the low-loss region of EEL spectra, 
therefore we will not go into more detail regarding the core ionisation peaks.


\subsection{Energy resolution}
The energy resolution of an EEL spectrum is determined by several factors. 
%
Firstly, aberrations of the electron spectrometer cause blur of the incoming 
electron beam~\cite{Freitag:2005}. 
%
In general, the spectrometer dispersion becomes worse for higher electron losses,
and therefore the resolution at ionization edges suffers more than close to the
ZLP.
%
Imaging quality can be improved by the implementation of an aberration corrector, 
to cancel the spherical aberration of the objective and condensor lenses. 
%
The energy resolution is then mainly determined by the angular distribution 
provided by the electron beam, usually expressed as the full width at half maximum (FWHM) of the ZLP. 
%
The peak width depends strongly on the electron source. 
An  electron  microscope  equipped  with  a  cold
field emission gun typically has an energy resolution of about 300-800 meV 
under normal operation conditions.  
%
While this width is small compared to the operating voltage of the STEM (usually between 60-300 keV), 
it sets a limit for the energy resolution of EELS and thereby hinders the ability to distinguish 
between peaks separated by less than this value. 
%
Furthermore, for low-loss phenomena such as excitons and intra-band transitions, 
excitation probabilities can be quite low and these signals can be lost in the tails of the ZLP.
%
The resolution can be drastically improved by implementation of a monochromator 
in the TEM. 
%
In monochromators, a small magnetic prism and energy-selecting slit are installed 
directly after the electron source.
%
This setup essentially works as an energy filter: the incoming electron beam is first dispersed 
before going through a narrow slit, restricting the energy distribution of the incoming electrons. 
%
After compressing it back into the electron probe, the width of the electron beam 
and the tail intensity are greatly reduced.  
%
In a recent studies~\cite{Krivanek:2009}, a monochromated zero-loss peak was obtained 
with a FWHM as small as 4.2 meV and, maybe even more importantly, 
the tail intensity at 20 meV loss has dropped below $10^{-3}$ of its maximum, 
allowing features in the very low-loss region to be resolved. 
%
The improving energy resolution opens new possibilities for accurate measurements
on the bandgap and the 
dielectric function.

Apart from the increase in resolution, another advantage of a monochromated 
electron beam is its symmetric energy distribution. 
%
This directly implies that asymmetries of absorption peaks in the spectrum can 
unambiguously be attributed to the response of the material~\cite{Erni:2005}.
%
The reduction of the energy spread of the incident electron beam often 
comes at the expense of the beam intensity. 
%

In Fig~\ref{fig:monochromation}, one can observe the effect of a monochromator 
on the zero-loss peaks of a Schottky field emission microscope in the work of~\cite{Erni:2005}. 
%
Due to thermal broadening, the unfiltered Schottky field-emission source
shows a broad tail at the high-energy side, {\it i.e.} at negative energy loss values.
%
The tails of the monochromated beam are highly symmetrical and the energy dispersion (FWHM)
is greatly reduced.

\begin{figure}[H]
    \centering
    \includegraphics[width=80mm]{plots/monochromator.png}
    \caption{Comparison of the zero-loss peaks of an unfiltered Schottky field emission microscope 
    (200 kV), a cold field-emission microscope (100 kV) and a monochromated electron beam with 
    Schottky field-emission source (200 kV). 
    The second ZLP shows a wider tail at the low-energy side and the tails of the 
    monochromated beam are highly symmetrical. 
    The FWHM indicated in each case provides a measure for the energy resolution.
    %
    Retrieved directly from~\cite{Erni:2005}.}
    \label{fig:monochromation}
\end{figure}


\subsection{ZLP subtraction}

The most important aspect complicating the study of the low-loss regime of EEL spectra 
is the observation of the zero-loss peak, a very intense and ubitiquous feature 
whose right-hand tail suppresses the low-loss features, which results in the loss
of important information within the spectrum.
%
Before analysing the low-loss region of EEL spectra, accurate removal of the
ZLP is crucial. 
%
In the last several years different removal routines for the ZLP were introduced, 
as the increasing energy resolution of the instrumentation would allow for bandgap 
determination by means of VEELS. 
%
The most general suggestion is the subtraction of a fitted ZLP rather than 
using the vacuum recorded ZLP, because a separately recorded peak is always 
different in shape than the one recorded on a sample, due to phonon excitations, 
exciton losses, and broadening at the surface of the sample.
%
Due to the obvious difference in shape between vacuum and in-sample recorded ZLPs, 
direct subtraction of the first on the latter would introduce an extra
source of error. 
%
For this reason, using a fitted ZLP and subtracting it from the EEL spectrum 
is the preferred way to go.
%
Representative examples include the subtraction of a fitted Lorentzian distribution~\cite{Dorneich:1998},
directly subtracting the mirrored left-hand side of the ZLP~\cite{Lazar:2003},
the subtraction of a power-law fit~\cite{Erni:2005}, and the use of a
more general multi-parameter function~\cite{Benthem:2001}.
%
These and several other attempts to model the ZLP distribution 
have had some success at describing the main intensity of the peak, 
but in the tails discrepancies can be as large as several tens of percent~\cite{Bangert:2003}.
%
The standard method for background subtraction of the tails
is to fit a power law to the tails, however this approach is not suitable in
many circumstances~\cite{Hachtel:2018, Tenailleau:1992, Reed:2002, Bosman:2006}.
%
Especially in the very-low-loss region, a simple functional fit completely
removes all intensities belonging to losses within the bandgap energy.
%
More recent studies use integrated software applications for background subtraction 
methods~\cite{Egerton:10.1016/S0304-3991(01)00155-3, Held:2020, Granerod:2018, Fung:2020}.\\

One common flaw of these subtraction methods is the fact that they are often based on specific
model assumptions about the ZLP properties and thereby introduce a methodological
bias which size is difficult to quantify. 
%
This bias arises from assumptions made {\it e.g.} on its functional form, symmetry 
properties, or the fitting range that has been used, all introducing an arbitrariness
to the procedure.
%
Even more importantly, these subtraction methods lack an estimate of the associated uncertainties, 
which in turn affects the reliability of any physical interpretation of features that are observed
in the ZLP-subtracted spectra. 

Developing a model-independent strategy that allows for a determination of the ZLP
with a faithful uncertainty estimate is highly coveted.
%
With the knowledge that the magnitude and shape of the ZLP depend
not only on the specific values
of the electron energy loss $\Delta E$, but also on other operation parameters
of the TEM such as the electron beam energy $E_{\rm b}$, the exposure time
$t_{\rm exp}$, the aperture width and the potential use of a monochromator,
one cannot measure the ZLP for a given operating
condition, for instance a high beam voltage of $E_{\rm b}=200$ keV, and expect to reproduce
the ZLP distribution
associated to different conditions, such as a lower beam voltage of $E_{\rm b}=60$ keV,
without introducing model assumptions.
%
Since it is not possible to compute the dependence of the ZLP on $\Delta E$
and the other operating conditions of the microscope from first principles,
reliance on specific models appears to be unavoidable.
%
Furthermore, even for identical operating conditions, 
the intensity of the ZLP will in general vary due to {\it e.g.} external perturbations 
such as electric or magnetic fields~\cite{Rafferty:2000},
the stability of the microscope and spectrometer electronics~\cite{Kothleitner:2003}, 
the local environment (possibly exposed to mechanical, pressure and temperature fluctuations) 
and spectral aberrations~\cite{Egerton:1996}. 
%
Any model for the ZLP should thus account for this source of uncertainties.


\newpage

% Review of TMDs and WS2
\section{Transition metal dichalcogenides and WS$_2$}
\label{sec:tmd}

In this work we will apply our machine learning-based method
for describing the ZLP to a novel class of tungsten disulfide (WS$_2$) nanostructures known
as nanoflowers~\cite{SabryaWS2}.

WS$_2$ is a transition metal dichalcogenide (TMD) material, which 
belongs to a large family of materials known as 2D materials or van der Waals materials.
%
They are named two-dimensional to emphasize their extraordinary thinness: 
TMDs are characterised by the remarkable property of being fully 
functional down to a single atomic layer.
%
A monolayer MoS$_2$ is only 6.5 Å thick. 
%
Just like graphite is a stacking of individual layers of graphene,
bulk crystals can be formed by stacking 2D monolayers, 
 bound to each other by Van der Waals attraction. 
%
Over the past few years the exploration of these 2D layered materials
 has developed rapidly. 
 %
 In particular significant attention has been 
 going to monolayers of transition metal dichalcogenides,
 atomically thin semiconductor of the type $MX_2$, here M is a 
transition metal atom (such as Mo or W) and X is a chalcogen atom (such as S, Se, or Te). 
 %
The characteristic crystalline structure of TMDs is one layer of M atoms 
that is sandwiched between two layers of X atoms.
 %
The electronic structure of TMDs strongly depends on the coordination 
 of the transition metal atoms, giving rise to a variety of electronic
 and magnetic properties~\cite{Chhowalla:2013}.
 %
In fact, most of the remarkable electronic and optical properties of TMDs
can be traced back to the underlying periodic arrangements of their layers, 
the so-called stacking sequences~\cite{SabryaWS2}.
%
Furthermore, the properties of this class of materials vary significantly
with their thickness, for instance MoS$_2$ exhibits an indirect bandgap
in the bulk form which becomes direct at the monolayer level~\cite{Splendiani:2010}.
%
The indirect-to-direct bandgap transition is the main reason for the interest in 
the use of TMDs for flexible electronics: it emphasizes the importance of the
mechanical properties of these materials. 
%
However, it tends to be much more difficult to uniformly deform 2D monolayers
of a material compared to bulk samples, and therefore measuring on 2D systems
can be challenging.
%
TMDs are often combined with other 2D materials like graphene
to make Van der Waals heterostructures, which need to be tuned in order
to function as building blocks for many devices such as LEDs, solar cells, 
transistors, and photodetectors.
%
This research field is still emerging and highly promosing to have a big
impact on future nanotechnology. \\
 
An example of a TMD exhibiting a pronounced dependence on its thickness is 
thungsten disulfide (WS$_2$), with an indirect-to-direct bandgap transition when going
from bulk to bilayer or monolayer form.
%
The effects of this transition are manifested as enhanced
photoluminescence in monolayer WS$_2$, whereas only little emission is observed in
the corresponding bulk form.
%
WS$_2$ adopts a layered structure by stacking atomic layers of S-W-S 
in a sandwich-like configuration. 
%
Although the interaction between adjacent layers is a weak Van der Waals 
force, the dependence of the interlayer interaction on the stacking 
order of WS$_2$ is significant.
%
Further applications of this material include storage of hydrogen 
and lithium for batteries~\cite{Bhandavat:2012}.

%%%%%%%%%%%%%%%%%%%%%%%%%%%%%%%%%%%%%%%%%%%%%%%%%%%%%%%%%%%%%%%%%%%%%%%%%%%%%%%
\begin{figure}[h]
    \centering
    \includegraphics[width=0.8\textwidth]{plots/spectrumimage.pdf}
    \caption{Left: low-magnification TEM image of WS$_2$ nanoflowers
      grown on top of a porous TEM substrate. Right: the magnification of a representative 
      petal of a nanoflower, where the black region corresponds to 
      the vacuum (no substrate) and the difference in contrast indicates terraces of varying thickness.}
    \label{fig:nanoflowers}
\end{figure}
%%%%%%%%%%%%%%%%%%%%%%%%%%%%%%%%%%%%%%%%%%%%%%%%%%%%%%%%%%%%%%%%%%%%%%%%%%%%%%%%%%5

A low-magnification TEM image of the WS$_2$ nanoflowers presented in~\cite{SabryaWS2} is displayed
in the left panel of Fig.~\ref{fig:nanoflowers}.
%
These nanostructures are grown directly on top of a TEM substrate with holes in it. 
%
The right panel shows the magnification of a representative petal of a nanoflower,
where the difference in contrast indicates terraces of varying thickness.
%
Note that the black region corresponds to the vacuum, without
substrate underneath.
%
These WS$_2$ nanoflowers contain areas with different thicknesses, orientations
and crystalline structures, therefore representing an ideal environment to investigate
structural morphology in WS$_2$ with electronic properties at the nanoscale.
%
What makes it even more interesting is that these nanoflowers display 3R/2H polytypism, 
an important issue for the interlayer
 interactions within WS$_2$: different stacking types tend to coexist, 
 complicating the characterization of the physical properties~\cite{Na:2018}.
%
One possible response of polytypism to electric fields is
 spontaneous electrical polarization, leading to modifications on the 
 electronic band structure and correspondingly on the band gap.
 %
Tailoring the specific stacking sequences (polytypes) 
represents a powerful strategy to identify and design novel physical properties~\cite{SabryaWS2}.
%

As mentioned before, one of the most interesting properties of TMDs that also
occurs in WS$_2$ is the fact when the material
is thinned down to a single monolayer, its indirect band gap of
$E_{\rm bg}\simeq 1.4$ eV
switches to a direct band gap of approximately $E_{\rm bg}\simeq 2.1$ eV.
%
In general, it has been found that the type and magnitude of the bandgap
of WS$_2$ depends quite sensitively on the crystalline structure and
the number of layers that constitute the material.
%
In Table~\ref{table:bgvalues} we collect
representative results for the determination of the bandgap energy $E_{\rm bg}$
and its type in WS$_2$, obtained by means of different experimental and theoretical techniques.
%
For each reference we indicate separately the bulk results and those
obtained at the monolayer level.
%
We observe that for monolayers, the results for the measured
value of $E_{\rm bg}$ are quite inconsistent, 
reflecting the challenges of its accurate determination.


 
%%%%%%%%%%%%%%%%%%%%%%%%%%%%%%%%%%%%%%%%%%%%%%%%%%%%%%%%%%%%%%%%%%%%%%%%%%%%%%%%%%%%%
\begin{table}[H]
  \small
  \begin{centering}
   \renewcommand{\arraystretch}{1.20}
\begin{tabular}{ccccc}
\br
Reference                       & Thickness & $E_{\rm bg}$ (eV)  & Band gap type  & Technique \\
\mr
{\cite{Braga:2012}} & bulk   & $1.4\pm0.07$            & indirect  & {Gate-voltage dependence}  \\
\mr
\multirow{}{}{\cite{Jo:2014}}                 & ML   & $2.14 $         & direct  & \multirow{}{}{Gate-voltage dependence}        \\
& bulk & $1.40 $    & indirect              \\
\mr

\multirow{}{}{\cite{Gusakova:2007}} & ML   & $2.03\pm0.03$            & direct  & \multirow{}{}{DFT}  \\
& bulk & $1.32\pm0.03 $            & indirect     \\
\mr
\multirow{}{}{\cite{Kam:1982}}                  & ML   & $1.76\pm0.03 $      & direct    & \multirow{}{}{Absorption edge coefficient fitting}         \\
& bulk & $1.35 $          & indirect        \\
\mr
\cite{Shi:2013}                & ML   & $2.21\pm0.3 $         & direct  & Bethe-Salpeter equation (BSE)        \\                 \br                                         
\end{tabular}
\vspace{0.27cm}
\caption{Representative results for the determination of the bandgap energy $E_{\rm bg}$
  and its type in WS$_2$, obtained from a variety of experimental and theoretical techniques.
  %
  For each reference we indicate separately the bulk results and those
  obtained for a monolayers}
    \label{table:bgvalues}
    \end{centering}
\end{table}
%%%%%%%%%%%%%%%%%%%%%%%%%%%%%%%%%%%%%%%%%%%%%%%%%%%%%%%%%%%%%%%%%%%%%%%%%%%%%%%%%%%%%%

\newpage

% Fundamentals of NN 
\section{Fundamentals of neural networks}
\label{sec:nn}
One of the most successful machine learning techniques, 
artificial neural networks, are based on the idea of simulating 
the functioning of neuron connections of the human brain. 
%
This machine learning technique is trained in a fashion similar to 
human learning with the goal to process complex inputs and conclude correct outputs \cite{greplova}.
%
A neural network (NN) is defined by a (usually large) number of neurons 
interconnected with strength parameters called weights. 
%
A typical multilayer neural network architecture is schematically shown below in Fig.~\ref{fig:nn}.
%

\begin{figure}[H]
    \centering
    \includegraphics[width=100mm]{plots/nn.png}
    \caption{Schematic representation of a four-layer Neural Network with two hidden layers and one output neuron}
    \label{fig:nn}
\end{figure}

%
Each neuron, represented in the above figure with a circle, 
is connected to a number of neurons in the previous layer.  
%
The far left neurons are the input neurons, they represent an input vector where 
each neuron holds one dimension of the vector, {\it e.g.} an energy loss bin in an EEL spectrum. 
%
The far right neurons are the output neurons and they can be be either
real-valued numbers or classification labels (1 or 0).
%
For each neuron, its output is the result of an activation function to its inputs:

\begin{equation}
    z = \sum_j w_j x_j + b,
\end{equation}

where $x_j$ are the outputs of the preceding neurons, 
$w_j$ are the corresponding weights and $b$ is the bias (offset) of the neuron. 
%
It is the latter two that will be optimized by training.
A nonlinear activation function $f(z)$ is applied to come to the output value for each neuron; 
this procedure is called forward propagation. 
%
Through the use of these nonlinear functions, the neural network graph 
resembles an efficient nonlinear regression model.
% 
Typical examples for $f(z)$ are the Rectified Linear Unit (ReLU) 
and sigmoid function, as depicted in Fig. \ref{activation}. 
%
Note that these are two often-used, but not the only possible activation functions. 
%
The choice of nonlinear activation function influences computational and 
training properties of the neural nets \cite{juan}.
%
For example, choice for the ReLU function ensures absolute positivity. 

\begin{figure}[H]
    \centering
    \includegraphics[width=100mm]{plots/f(z).png}
    \caption{Sigmoid (left) and ReLU (right) activation functions. Both approach 0 as $z\rightarrow -\infty$ but note the different behaviour for $z\gg0$. }
    \label{activation}
\end{figure}

The input vector is passed through each layer of the neural network and is transformed
by the nonlinear operations until it arrives at the output layer.
%
For a certain set of inputs \textbf{$x_0$} and a collection of weights and biases $(w^{\rm l}_{\rm jk}, b^{\rm l}_j)$, 
where $w^{\rm l}_{\rm jk}$ is the connection between the j-th neuron in the l-th layer and the k-th 
neuron in the (l+1)-th layer, $b^{\rm l}_{\rm j}$ being the corresponding bias,
the neural network gives an output $y$ that can be compared to the desired (target) output $\hat{y}$. 
%
A cost function $C(y, \hat{y})$
is used to quantify how far we are from our desired output.
%
It is for this reason that this type of machine learning is called supervised learning: after
each iteration, under supervision of the correct output, the model parameters are adjusted.
%
This is in contrast to unsupervised learning, 
where one only has a collection of data, and is looking to find structure without knowing the desired outcome.

Training the algorithm is done by minimizing the cost function with respect to all tunable parameters
$(w^{\rm l}_{\rm jk}, b^{\rm l}_j)$ of the system. 
%
This minimization is done by means of the gradient descent method, a first-order optimization algorithm
for finding a local minimum of a differentiable function, in this case the cost function $C$.
%
To find the local minimum, after each iteration the weights and biases are adjusted a bit proportional
to the negative of the gradient of the cost function at the current point. 
%
To make this minimization computationally tractable, 
the so-called back-propagation algorithm is used \cite{Hecht:1992}, 
allowing us to approximate the derivative of the cost function by averaging over the training set. 
%
All together, training the network relies on the following four fundamental equations:

\begin{align}
\label{eqs3}
\begin{split}
 \delta_j^L &= \frac{\partial C}{\partial a_j^L} f'(z_j^L), 
 \\
 \delta^l &= (w^{l+1} \delta^{l+1}) \odot f'(z^l),
 \\
 \frac{\partial C}{\partial b_j^l} &= \delta_j^l,
 \\
 \frac{\partial C}{\partial w_{jk}^l} &= a_k^{l-1}\delta_j^l.
\end{split}
\end{align}

Here, $\delta_j^l$ is the error and $a_j^l$ is the output of neuron $j$ in layer $l$; 
$C$ is the cost function; $b$ is the bias and $w$ is the weight of each neuron. 
%
It is the error $\delta_j^L$ in the first line that represents the total cost of the network.
%
The second line presents the backpropagation: from layer $(l+1)$, the error is propagated
back through the network until the very first layer.
%
The last two equations evaluate the gradient of the model parameters 
with the gradient of the cost function. 
%
With this info one can update the model parameters by

\begin{align}
\begin{split}
 b^{l+1} &= b^l - \eta \frac{\partial C}{\partial b^l}
 \\
 w^{l+1} &= w^l - \eta \frac{\partial C}{\partial w^l}
\end{split}
\end{align}
where $\eta$ is the pre-defined learning rate, a measure for the size 
of the step taken with each iteration.
%
From Eqns.~(\ref{eqs3}), one can see that the choice of nonlinear 
activation function $f(z)$ affects the learning of the network. 
%
This could already be deduced from the shape of the functions in Fig.~\ref{activation}: 
since $\sigma$(z) saturates for large inputs $z\gg0$, 
$d\sigma/dz\rightarrow0$ and the network loses sensitivity~\cite{juan}.


We use the definition {\it epoch} for each time that
the entire set of training data is passed forward once through the neural network 
and the network has backpropagated the error and updated all of its parameters
accordingly. 
%
The network needs much more than just one epoch to optimize by means of 
gradient descent, as this is an iterative process. 
%
After each epoch, the performance of the network is evaluated by calculating
the error ($ \delta_j^L$) on the training data and the weights and biases
are adjusted accordingly.
%
As the number of epochs increases, the network parameters are adjusted
repeatedly and where the network was first underfitting the inputs at
the beginning, at a certain moment it goes to optimal fitting,
before it enters the overfitting regime.
%
Several methods can be applied to determine the optimal stopping point
of the network, that is, to find the moment at which the network is
neither under-, nor overfitting the training data. 
%
One of such is splitting the total set of experimental data into two sets:
the training dataset is the one we use to train the model, usually
80\% of the total set.
%
The other 20\% is what we call the validation set and it is used to provide
an unbiased evaluation of the model fit on the training set. 
%
This split ratio is common for models with a moderate number of hyperparameters:
increasing the size or increasing the tunability of the model usually goes with increasing the share of the
validation set. 
%
The validation subset is left out of the training set on purpose and the model
can not learn on these inputs.
%
After each epoch, the total performance of the
system can be validated by feeding this subset to the network and calculating
the total corresponding error.
%
Tracing the cost function on both the training and the validation set 
gives insight in if the network is overfitting the training data.
First, both the training and validation error will decrease, 
but at a certain point the network will start overfitting and the
validation error slowly starts to increase. 
%
The optimal stopping point is defined as the global minimum of the 
error of the validation sample, computed over a large fixed number of 
iterations.
%
A typical progress of the training and validation error over the 
course of the optimization can be observed in Fig.~\ref{fig:costs}.

\begin{figure}[H]
\centering
\includegraphics[width=150mm]{plots/train_val_error.pdf}
\caption{Progress of both the training and validation error over 
one training session as a function of the number of epochs. 
The optimal stopping point is where the validation error is at
its absolute minimum, here after 43,500 epochs.}
\label{fig:costs}
\end{figure}

%
Once the optimal network parameters have been determined and stored, 
the network can be used to
make predictions on any set of inputs. 







\newpage

% Discuss the methodology that we are going to use
%%%%%%%%%%%%%%%%%%%%%%%%%%%%%%%%%%%%%%%%%%%%%%%%%%%%%%
\section{Neural network determination of the ZLP}
%%%%%%%%%%%%%%%%%%%%%%%%%%%%%%%%%%%%%%%%%%%%%%%%%%%%%
\label{sec:methodology}

In this section we present our strategy to parametrise, predict and subtract 
the zero loss peak by means of machine learning.
%
As mentioned in the introduction, our strategy will be inspired by the 
NNPDF method~\cite{Rojo:2018qdd} originally developed in the context of high-energy physics
for studies of the quark and gluon substructure of the proton~\cite{Gao:2017yyd}.
%
The NNPDF approach has been successfully applied, among others, to
the determination of
unpolarised~\cite{DelDebbio:2007ee,Ball:2008by,Ball:2012cx,Ball:2014uwa,Ball:2017nwa}
and polarised parton distributions functions of protons, nuclear
parton distributions~\cite{AbdulKhalek:2019mzd,AbdulKhalek:2020yuc}, and the
fragmentation functions of partons into neutral and charged
hadrons~\cite{Bertone:2017tyb,Bertone:2018ecm}.
%
Neural networks benefit from the ability to parametrise 
multidimensional input data with arbitrarily non-linear dependencies:
even with a single hidden layer, a neural network can reproduce arbitrary 
functional dependencies provided it has a large enough number of neurons.
%
We can therefore apply a similar procedure for the determination of the
functional dependence of the ZLP intensity. 

We note that recently several applications of machine learning
to transmission electron microscopy analyses 
in the context of material science have been
presented, see {\it e.g.}~\cite{Gordon:2020, Zhang:2019, Jany:2017, Ziatdinov:2017,10.1145/2834892.2834896,doi:10.1021/acsnano.7b07504,cite-key}.
%
Representative examples
include the automated identification
of structural information at the atomic scale~\cite{10.1145/2834892.2834896} 
and the extraction of chemical information
and defect classification~\cite{doi:10.1021/acsnano.7b07504}.
%
For the readout of EEL spectra specifically, 
machine learning has been put forward for the prediction
of spectral features in the core-loss regime~\cite{Kiyohara:2018}.
%
To the best of our knowledge, this is the first time that neural networks are used as 
unbiased background-removal interpolators and that they are used combined with 
Monte Carlo sampling to construct an estimate of the model uncertainties.\\

In this section, we discuss the parametrisation of the ZLP in terms of neural networks.
%
The ultimate goal is to create a model that is able to predict the contribution $I_{\rm ZLP}$
in the total intensity profile of any EEL spectrum recorded over a specimen, 
and subsequently to subtract this distribution from the spectrum to isolate the inelastic
scattering contributions.
%
In order to do so, we first need to develop a model that is able to predict the general
shape of the zero loss peak as a function of its input parameters. 
%
For this we use in-vacuum zero loss peak recordings, which function as a baseline to 
create this generic, multidimensional model.
%
In this regard we also
explain the Monte Carlo replica method, which is used to estimate and propagate the
uncertainties from the input data to the model predictions.
%
After this we move on to the training strategy on sample spectra: 
we explain how the method is modified to use it on spectra recorded over
WS$_2$ specimens and how one can select the hyper-parameters that appear in the model.


\subsection{ZLP parametrisation}
\label{sec:parametrisation}

Without any loss of generality, we can decompose recorded the intensity profile
in any EEL spectrum as
\be
\label{eq:IeelTot}
I_{\rm EEL}(\Delta E) =I_{\rm ZLP}(\Delta E) + I_{\rm inel}(\Delta E) \, ,
\ee
where $\Delta E$ is the measured electron energy loss; $I_{\rm ZLP}$ is the zero loss
distribution arising both from instrumental origin and from elastic interactions; and
$I_{\rm inel}(\Delta E)$ contains the contributions from the electrons that have undergone
inelastic scattering with the specimen. 
%
It is the latter contribution that we're particularly interested in, but in order 
to get hold of it we need to disentangle it from the zero loss contribution.
%
As shown by the representative example of Fig.~\ref{fig:EELS}, there are two limits
for which one can straightforwardly separate the two contributions.
%
The first region is for sufficiently high energy losses, where
$I_{\rm ZLP}$ vanishes and $I_{\rm EEL} \to I_{\rm inel}$.
%
Secondly, in the region close to zero, all emission can be associated to
the ZLP such that $I_{\rm EEL}\to  I_{\rm ZLP}$.
%
It is the region in between that is of particular interest, 
the ultra-low-loss region where $I_{\rm ZLP}$ and $I_{\rm inel}$
become of comparable order of magnitude.

Our goal is to construct a parametrisation of $I_{\rm ZLP}$ based on artificial
neural networks, which we denote by $I_{\rm ZLP}^{\rm (mod)}$, which allows us to
extract the relevant inelastic contribution by subtracting the
contribution of the ZLP from the total EEL spectra:
\be
\label{eq:ZLPseparation}
I_{\rm inel}(\Delta E) \simeq I_{\rm EEL}(\Delta E) - I_{\rm ZLP}^{\rm (mod)}(\Delta E) \,.
\ee
Isolating $I_{\rm inel}$ from the total spectrum makes us able to exploit 
the physical information contained in the low-loss region.
%
Crucially, we aim toestimate and propagate all the relevant sources of uncertainty associated
both to the input data and to methodological choices. 
%
This helps us to verify our results and to separate lucky findings from real insights.\\

As discussed in Sect.~\ref{sec:eels}, the ZLP depends both
on the value of the electron energy loss $\Delta E$ as well as on the operating
conditions of the microscope, such as the electron beam energy $E_b$ and the exposure time
$t_{\rm exp}$.
%
Therefore we want to construct a multidimensional model which can theoretically take any number of relevant variables
as input, in order to reproduce the predicted zero loss peak.
%
This means that in general Eq.~(\ref{eq:ZLPseparation}) can be written as
\be
I_{\rm inel}(\Delta E) = I_{\rm EEL}(\Delta E, E_{b},t_{\rm exp}, \ldots) - I_{\rm ZLP}^{\rm (mod)}(\Delta E, E_{b},t_{\rm exp}, \ldots) \, ,
\ee
where we note that the subtracted spectrum $I_{\rm inel}(\Delta E)$ should depend only on the energy loss, but not on the microscope parameters.
%
Ideally, the ZLP model should be able to accomodate as many input variables as possible.
%
The output of the ZLP model is parametrised by means of multi-layer feed-forward artificial neural networks.
This means that the predicted zero loss peak intensity can be expressed as 
\be
\label{eq:ZLPmodelNN}
I_{\rm ZLP}^{\rm (mod)}(\Delta E, E_{b},t_{\rm exp}, \ldots)  = \xi^{(n_l)}_1(\Delta E, E_{b},t_{\rm exp}, \ldots) \, ,
\ee
where $\xi^{(n_l)}_1$ is the activation state of the single output neuron in the last
of the $n_l$ layers of the network. Here the $n_I$ inputs are the variables $\{ \Delta E, E_{b},t_{\rm exp}, \ldots \}$
that represent the relevant information about the operating conditions during the recording of the spectra.
%
Note that for this work, we have used spectra recorded under the known conditions $\{ \Delta E, E_{b},t_{\rm exp}\}$. 
%
This set of inputs could potentially be extended by including extra variablesm, 
{\it e.g.} arperture width, abberation correction and temperature. 
%
The neural network is then trained by means of supervised learning and non-linear regression on these  
$n_I$ inputs, using the known corresponding ZLP intensities $I_{\rm ZLP}^{\rm (mod)}$ as training outcomes. 
%
Each iteration, the weights and thresholds of this neural network model are optimized
from the minimization of the error on this training dataset.\\

The number of hidden layers and neurons that is optimal is very task depenend and
should therefore be decided arbitrarily, 
there is no general rule of thumb. 
%
We have chosen to use an $n_I$-10-15-5-1 architecture with three hidden layers, wich corresponds to a total
number of 289~(271) free parameters for $n_I=3$~($n_I=1$) to be optimised.
%
However, we have verified that results are fairly independent of this exact choice:
predictions on the training data did not change significantly when the architecture 
was increased by a factor of two.

  
%%%%%%%%%%%%%%%%%%%%%%%%%%%%%%%%%%%%%%%%%%%%%
\begin{figure}[t]
    \centering
    \includegraphics[width=99mm]{plots/architecture.pdf}
    \caption{Schematic representation of our ML model for the ZLP, Eq.~(\ref{eq:ZLPmodelNN}).
      %
      The input is an $n_I$-dimensional array containing $\Delta E$ and other
      operation variables of the microscope such as $E_b$ and $t_{\rm exp}$.
      %
      The output is the predicted value of the intensity of the zero-loss peak
      distribution associated to those specific input variables.
      %
      The architecture is chosen to be $n_I$-10-15-5-1, with sigmoid activation functions
      in all layers except for a ReLU in the output neuron.
    }
    \label{fig:architecture}
\end{figure}
%%%%%%%%%%%%%%%%%%%%%%%%%%%%%%%%%%%%%%%%%%%%%%%%%

A schematic representation of this model
is displayed in Fig.~\ref{fig:architecture}.
%
 The input is an $n_I$ array containing $\Delta E$ and the rest of
 operation variables of the microscope, and
 the output is the value of the intensity of the ZLP distribution
 associated to those input variables.
 %
 We use a sigmoid activation function for the three hidden layers and a ReLU
 for the final one.
 %
 The choice of ReLU for the final layer guarantees that our model for the ZLP
 is positive-definite, as required by general physical considerations: the intensity
 count can never be smaller than zero.
 %
 We have adopted a redundant architecture to ensure that the ZLP parametrisation
 is sufficiently flexible, which means that this way we guarantee that
 the network can over-fit on the training inputs.
 %
 However, the final results should be evaluated before the network starts overfitting,
 as described in Sect.~\ref{sec:neuralnetworks}. 
 %
 This means that we need to define a suitable regularisation strategy, which 
 will be explained later onwards in Sect.~\ref{sec:training}.



\subsection{Uncertainty propagation}
\label{sec:uncertaintypropagation}

We discussed in Sect.~\ref{sec:eels} how
even for EEL spectra taken at identical operating conditions of the microscope,
in general the resulting ZLP intensity profiles will be different.
%
Also, the input data can be described by a large number of different neural 
network configurations, each with a different functional form of $I_{\rm ZLP}^{(\rm mod)}$
but representing the data equally well.
%
The Monte Carlo replica method can be used to estimate these two sources of 
uncertainties, introduced by the experimental data and the methodology,
and to propagate them to physical predictions.
%
The basic idea is twofold:
first, it is useful to represent problems with a possibility of non-gaussian errors
through the use of their central values and uncertainties, which are obtained from a Monte Carlo sample
as their averages and standard deviations.
%
Second, when a problem requires a reconstruction of discrete sampling without making assumptions on its 
functional form, neural networks are useful to work as unbiased interpolators. 
%
It is the combination of both techniques that explains the use of neural networks to separate a smooth
signal from background signals, while the MC samples handles the fluctuations within the data.

%%%%%%%%%%%%%%%%%%%%%%%%%%%%%%%%%%%%%%%%%%%%%%%
\begin{figure}[h]
    \centering
    \includegraphics[width=0.7\textwidth]{plots/MCscheme.pdf}
    \caption{Representation of the Monte Carlo replica strategy. From the original set of training data,
    an ensemble of replicas is created from the experimental central values and uncertainties. 
    On each replica, an individual neural network is trained and a ZLP parametrisation is obtained.
    The total set of predictions is then used to calculate physical observables from the corresponding
    central values and uncertainties.}
    \label{fig:MCscheme}
\end{figure}
%%%%%%%%%%%%%%%%%%%%%%%%%%%%%%%%%%%%%%%%%%%%%%%%5

The MC strategy is schematically summarized in figure \ref{fig:MCschema} and it involves two stages.
In the first, we generate an ensemble of replicas of the original training set. 
%
Then, each replica
is passed through an individual neural network, which thereby outputs a parametrisation of the ZLP.
%
Any physical observables can be calculated over the set of computed ZLP distributuions.\\

Let us assume that we have $n_{\rm dat}$ independent measurements of the ZLP intensity, 
so our training dataset contains $n_{\rm dat}$ data points, all with different combinations of input parameters. 
%
The collective of inputs are given as $\{z_i\}$:
\be
I^{\rm (exp)}_{{\rm ZLP},i}\lp \{ z_i  \}\rp = I^{\rm (exp)}_{{\rm ZLP},i}\lp  \Delta E_i, E_{b,i}, t_{\rm exp,i},\ldots \rp
\,, \quad i=1,\ldots,n_{\rm dat} \, .
\ee

The Monte Carlo method is based on the generation
of a large number $N_{\rm rep}$ of Monte Carlo replicas of these original data points
by means of a multi-Gaussian distribution, where we use the central values and covariance matrices
from the input measurements. 
%
The strategy of the generation of Monte Carlo replicas goes as follows: we create for each original data point
($I^{\rm (exp)}_{{\rm ZLP},i}$) an ensemble of artificial pseudo points (replicas): 
\be
\label{eq:MCreplicaGen}
  I_{{\rm ZLP},i}^{{\rm (art)}(k)}  =  I^{\rm (exp)}_{{\rm ZLP},i} + r_i^{({\rm stat},k)}\sigma_i^{\rm (stat)}
  + \sum_{j=1}^{n_{\rm sys}} r_{i,j}^{({\rm sys},k)} \sigma_{i,j}^{\rm (\rm sys)} \,, \quad \forall i
  \,, \quad k=1,\ldots,N_{\rm rep} \,.\,\, \,
  \ee
  where $\sigma_i^{\rm (stat)}$ and $\sigma_{i,j}^{\rm (\rm sys)}$ represent the statistical
  and systematic uncertainties (the latter divided into  $n_{\rm sys}$ fully point-to-point correlated
  sources) and $\{r_i^{(k)}\}$ are Gaussianly distributed random numbers.
  %
In the end, each $k$-th replica contains exactly
as many data points as the original set.

In our case we have no information on experimental correlations betweem the measurements and
for this reason we assume that there is only one single source of point-by-point systematic
uncertainty, which is uncorrelated. 
%
  Should in the future correlations became available, it would be straightforward to extend
  our model to that case.
%
In other words, to each data point we associate an individual uncertainty and we 
discard covariances between datapoints, which means that we drop the last term in Eq.~(\ref{eq:MCreplicaGen})
and it reduces to
\be
\label{eq:MCreplicaGen2}
  I_{{\rm ZLP},i}^{{\rm (art)}(k)}  =  I^{\rm (exp)}_{{\rm ZLP},i} + r_i^{({\rm stat},k)}\sigma_i^{\rm (stat)}
\ee

These statistical errors on the training data can be derived by means of what is called
equal width discretization (EWD) and it works as follows.
%
The input measurements will be composed in general on subsets of EEL
spectra taken with identical operation conditions.
%
For example, we have one specific set of $N_{sp}$ spectra all recorded with 
the same exposure time and beam energy. 
%
Since the range of $\Delta E$ over which the spectra have been recorded
is usually not the same in each case, first of 
we uniformise a common binning in $\Delta E$ with $n_{\rm dat}$ entries.
%
Then we evaluate the total experimental uncertainty in one of these bins as
\be
\sigma_i^{\rm (exp)} = \lp \frac{1}{N_{\rm sp}-1} \sum_{l=1}^{N_{\rm sp}}
\lp I_{{\rm ZLP},i}^{ ({\rm exp}),l}  - \la I_{{\rm ZLP},i}^{ ({\rm exp})}\ra_{N_{\rm sp}} \rp \rp^{1/2} \, ,\,
i=1,\ldots, n_{\rm dat} \, ,
\ee
that is, $\sigma_i^{\rm (exp)}$ is the standard deviation in bin $i$ calculated over the $N_{\rm sp}$ spectra.
%
This uncertainty is separately evaluated for each set of microscope operation conditions
for which data available.
%
The value of the number of generated MC replicas, $N_{\rm rep}$, should be chosen such that the set of replicas 
models accurately the probability distribution of original training data.
%
%%%%%%%%%%%%%%%%%%%%%%%%%%%%%%%%%%%%%%%%%%%%%%%
\begin{figure}[t]
    \centering
    \includegraphics[width=0.99\textwidth]{plots/MC.pdf}
    \caption{Comparison between the original experimental central values
      $I_{\rm ZLP,i}^{\rm exp}$ (left panel) and the corresponding statistical
      uncertainties $\sigma_i^{(\rm stat)}$ with the results of averaging over
      a sample of $N_{\rm rep}$ Monte Carlo replicas generated by means of
      Eq.~(\ref{eq:MCreplicaGen}), for different values of
      $N_{\rm rep}$.
      }
    \label{fig:MC}
\end{figure}
%%%%%%%%%%%%%%%%%%%%%%%%%%%%%%%%%%%%%%%%%%%%%%%%5

To verify that this is the case,
Fig.~\ref{fig:MC} displays a comparison between the original experimental central values
$I_{{\rm ZLP},i}^{\rm (exp)}$ (left) and the corresponding 
total uncertainties $\sigma_i^{(\rm exp)}$ (right panel) with the results of averaging over
a sample of $N_{\rm rep}$ Monte Carlo replicas generated by means of
Eq.~(\ref{eq:MCreplicaGen}) for different number of replicas.
%
We find that $N_{\rm rep}=500$ is a value that ensures that both
the central values and uncertainties are reasonably well reproduced,
and we adopt it in what follows.


\subsection{Training strategy}
\label{sec:training}

The training of the neural-network model for the ZLP peak differs between
the cases of EEL spectra taken on vacuum,
where by construction $I_{\rm EEL}(\Delta E) =I_{\rm ZLP}^{\rm (mod)}(\Delta E)$,
and for spectra taken on samples. 
%
In fact, EEL spectra taken in the vacuum but close
to the sample might still receive inelastic contributions coming from the specimen.
%
When we talk about vacuum spectra, we consider exclusively those taken 
reasonably far from the surface of the analysed nanostructures.
%
In the latter case, we need to find a training strategy to make predictions
about the ZLP distribution in the low-loss regime, while we can not directly
train on data in this region.
%
As indicated by Eq.~(\ref{eq:ZLPseparation}), in order to avoid
biasing the results it is important to ensure that the model is trained only on the region of the spectra
where the ZLP dominates over the inelastic scatterings.
%
Then we can make predictions in the extrapolation regime based on these 
training parameters.
%
We now describe the training strategy that is adopted in both cases.

\paragraph{Training of vacuum spectra.}
%
For each of the $N_{\rm rep}$ generated Monte Carlo replicas, we train an independent
neural network as described in Sect.~\ref{sec:parametrisation}.
%
Fitting the neural networks parameters to the data is performed by minimisation of a
figure of merit (error function), defined as:
\begin{equation}
  \label{eq:chi2}
\begin{centering}
  E^{(k)}\lp \{\theta^{(k)}\}\rp = \frac{1}{n_{\rm dat}}\sum_{i=1}^{n_{dat}}\left(\frac{ I_{{\rm ZLP},i}^{{\rm (art)}(k)} -
  I_{{\rm ZLP},i}^{{\rm (mod)}}\lp \{\theta^{(k)}\}\rp }{\sigma_i^{(\rm exp)}}\right)^2, 
\end{centering}
\end{equation}

which is the $\chi^2$ per data point comparing each replicated ZLP intensity
with the corresponding model prediction.
%
In this expression $E^{(k)}\lp \{\theta^{(k)}\}\rp$ is the error for the values 
$\{\theta^{(k)}\}$ of the network weights and thresholds;
$I_{{\rm ZLP},i}^{{\rm (art)}(k)}$ is the $k$-th replica for the ZLP 
intensity and $I_{{\rm ZLP},i}^{{\rm (mod)}}$ is the model prediction on this
replica; and ${\sigma_i^{(\rm exp)}}$ again represents the error
associated to this experimental data point.\\

The chi-squared method is the cornerstone of almost all fitting, 
as it is an intuitively reasonable measure of how well the predictions fit the data. 
%
If the model predictions are all within one standard deviation from the data, 
then the $\chi^2$ per data point takes a value roughly equal to 1. 
%
In general, if $\chi^2/n_{dat}$ is of the order 1, we can say that the 
fit is a good approximation to the real data. 

In order to speed up the neural network training process, prior to the optimisation
all inputs and outputs are scaled to lie between $[0.1, 0.9]$ before
being feed to the network.
%
This preprocessing facilitates that
the neuron activation states will typically
lie close to the linear region of the sigmoid activation function.\\

The contribution to the figure of merit from the input experimental data, Eq.~(\ref{eq:chi2}),
needs in general to be complemented with that of theoretical constraints on the model.
%
For instance, when determining nuclear parton distributions~\cite{AbdulKhalek:2020yuc}, one needs to
extend Eq.~(\ref{eq:chi2}) with Lagrange multipliers to ensure that both the $A=1$ proton boundary
condition and the cross-section positivity are satisfied.
%
In the case at hand, our model for the ZLP should satisfy the constraint that $I_{\rm ZLP}(\Delta E)\to 0$
when $|\Delta E| \to \infty$, since far from $\Delta E\simeq 0$ the contribution from the ZLP
is completely negligible.
%
In order to implement this constraint, we add $n_{pd}$ pseudo-data points to the training dataset 
"far" away from the ZLP and we modify
the figure of merit Eq.~(\ref{eq:chi2}) accordingly,
\be
\label{eq:chi2modified}
E^{(k)}\lp \{\theta^{(k)}\}\rp \to E^{(k)}\lp \{\theta^{(k)}\}\rp +
\lambda \sum_{i'=1}^{n_{pd}}\left(
  I_{{\rm ZLP},i'}^{{\rm (mod)}}\lp \{\theta^{(k)}\}\rp \right)^2, 
  \ee
  where $\lambda$ is a Lagrange multiplier whose value is tuned to ensure that the $I_{\rm ZLP}(\Delta E)\to 0$
  condition
  is satisfied without affecting the description of the training dataset.
  %
  In other words, we add a contribution to the error function coming from these pseudo points in the
  higher-loss regime, making sure that the optimization of the error function also takes into account
  the fact that at higher losses the intensity should vanish.
  %
  However, when calculating the overall fit quality after training, we do not include the second term
  on the right-hand side of Eq.~(\ref{eq:chi2modified}) in the evaluation of the figure of merit, 
  as these "fake" datapoints do not give any physical
  information about the original dataset whatsoever.

  The pseudo-data points are added in the region $\lc \Delta E_{\rm pd}^{\rm (min)},
  \Delta E_{\rm pd}^{\rm (max)}\rc$, and symmetrically for negative energy losses.
  %
The value of $\Delta E_{\rm pd}^{\rm (min)}$
can be determined automatically by evaluating the ratio $\mathcal{R}_{\rm sig}$ between the central
experimental intensity and the total uncertainty in each data point,
\be
\label{eq:pdlocation}
\mathcal{R}_{\rm sig}(\Delta E_i)\equiv \frac{I_{{\rm ZLP}}^{(\rm exp)}(\Delta E_i)}{\sigma^{(\rm exp)}(\Delta E_i)} \, ,
\ee
which corresponds to the statistical significance for the $i$-th bin of $\Delta E$ to differ from the null hypothesis
(zero intensity) taking into account the experimental uncertainties.
%
For sufficiently large energy loss this ratio approaches 1, which indicates that we are essentially
fitting statistical noise. 
%
In order to avoid this and only fit data that is different from zero within errors, we determine the value
of $\Delta E_{\rm pd}^{\rm (min)}$ from the condition that $ \mathcal{R}_{\rm sig}(\Delta E_i) \simeq$ 1. 
%
We keep the training data in the region $\Delta E \le \Delta E_{\rm pd}^{\rm (min)}$ and the pseudo-data
points are added for $\lc \Delta E_{\rm pd}^{\rm (min)}, \Delta E_{\rm pd}^{\rm (max)}\rc$. 
%
The value of $\Delta E_{\rm pd}^{\rm (max)}$ can be chosen arbitrarily and can be as large as necessary
to ensure that $I_{\rm ZLP}(\Delta E)\to 0$ as $|\Delta E| \to \infty$.

We note that another important physical condition on the ZLP model, namely its positivity
(since in EEL spectra the intensity is just a measure of the number of counts in the
detector for a given value of the energy loss) is automatically satisfied since
we use a ReLU activation function for the last layer.
%
A further obvious requirement is that the best fit is independent of any assumptions 
made about the ZLP distribution. 
%
This requirement can be met by making sure the parametrisation is redundant: 
the size of the neural network used, and therefore the number of optimizable parameters, 
is much larger than the minimum required in order to reproduce the data. 
%
This redundancy can be verified {\it a posteriori}, that results are independent of 
the size and architecture of the neural network.\\

In this work we adopt the {\tt TensorFlow} neural-net libraries to assemble
the architecture illustrated in Fig.~\ref{fig:architecture}.
%
Before training, all weights and biases are initialized in a non-deterministic order
by the built-in global variable initializer. 
%
The optimisation of the figure of merit Eq.~(\ref{eq:chi2modified}) is carried
out by means of stochastic gradient descent (SGD) combined with backpropagation. The
specific SGD optimizer used is the Adam algorithm.
%
The hyper-parameters of the optimisation algorithm such as the learning rate
have been adjusted to ensure proper learning is reached in the shortest amount
of time possible.
%
Given that we have a extremely flexible parametrisation, one should be careful
to avoid overlearning the input data.
%
Here over-fitting is avoided by means of a cross-validation stopping criterion.
%
We separate the input data into training a validation subsets, with a 80\%/20\% splitting
which partition varies randomly for each Monte Carlo replica.
%
We then run the optimiser for a very large number of iterations and store both
the state of the network and the value
of the figure of merit Eq.~(\ref{eq:chi2}) restricted to the validation
dataset, $E^{(k)}_{\rm val}$ (which is not used for the training).
%
We are not looking for the absolute minimum of the error function, 
but we rather search for an optimal stopping point to avoid overfitting. 
%
At this optimal stopping point, it reproduces the information contained in the dataset, 
but not its statistical fluctuations.
%
This point can be determined by the formulation of a stopping criterion, 
making it possible to, once the network completed the training, 
choose the parametrization of the network weights right before it 
entered the overlearning regime. 
%
The optimal stopping point is determined  for each replica
by keeping track of the figure of merit, which typically evolutes
as depicted in Fig.~\ref{fig:costs}. 
%
The specific network configuration that leads to the deepest minimum of $E^{(k)}_{\rm val}$:
is chosen {\it a posteriori}, that's why it is called look-back stopping, 
a method that has been widely used in the context
of neural networks.
%
The number of epochs should be chosen high enough to reach for each replica 
the absolute minimum of $E^{(k)}_{\rm val}$, 
rather than a local minimum.
For this work we need approximately 40,000 epochs to guarantee overlearning.
%
This corresponds to a serial running time of 60 seconds per replica when running the optimization on a 
single CPU for a set of 500 datapoints.
%
Once the training of all the $N_{\rm rep}$ neural network models for the ZLP has been carried
out as specified above, we quantify the overal fit quality of the model by computing the
$\chi^2$ defined as
\begin{equation}
  \label{eq:chi2_final}
\begin{centering}
  \chi^2 = \frac{1}{n_{\rm dat}}\sum_{i=1}^{n_{dat}}\left(\frac{ I_{{\rm ZLP},i}^{{\rm (exp)}} -
 \la I_{{\rm ZLP},i}^{{\rm (mod)}}\ra_{\rm rep} }{\sigma_i^{(\rm exp)}}\right)^2, 
\end{centering}
\end{equation}
which is the analog of Eq.~(\ref{eq:chi2_final}) now comparing the average model prediction
to the original experimental data values.
%
A value $\chi^2 \simeq 1$ indicates that a satisfactory description of the experimental data,
within the corresponding uncertainties, has been achieved.
%
Note that in realistic scenarios $\chi^2$ can deviate from unity, for instance when
some source of correlation between the experimental uncertainties has been neglected
or when the errors on the data points have been over- or underestimated.


\paragraph{Training of sample spectra.}

The training strategy in the case of EEL spectra taken on samples (rather than on vacuum) must be adjusted
to account for the fact that the input data set, Eq.~(\ref{eq:IeelTot}), receives contributions
both from the ZLP and from inelastic scatterings.
%
To avoid biasing the ZLP model, we should make sure to include only the former contributions 
in the training dataset.
%
Else, if we were to train the neural network on data that contains also inelastic contributions,
subsequent subtraction to the EEL spectra would make us lose important information.

We can illustrate the situation at hand with the help of a toy model for the low-loss
region of EEL spectra, represented in
Fig.~\ref{fig:EELS_toy}.
%
Let us use a general assumption that the ZLP is described by a 
Gaussian distribution with a standard deviation of $\sigma_{\rm ZLP}=0.3$ eV,
and that the contribution from the
inelastic interactions with the sample can be approximated in the low-loss
regime by $I_{\rm inel}(\Delta E)\propto \lp \Delta E - E_{\rm bg}\rp^b$ with $E_{\rm bg}=1.5$
and $b=0.5$. 
%
The motivation for this
choice will be explained in Sect.~\ref{sec:results_sample}.
%
We display the separate contributions from $I_{\rm ZLP}$
and $I_{\rm inel}$, as well as the total intensity,  
with the inset showing the values of the corresponding derivatives, $dI/d\Delta E$.

%%%%%%%%%%%%%%%%%%%%%%%%%%%%%%%%%%%%%%%%%%%%%
\begin{figure}[t]
    \centering
    \includegraphics[width=0.7\textwidth]{plots/Toy.pdf}
    \caption{A toy model for a generic EEL spectrum and its
      derivatives (in the inset).
      %
      We show the separate contributions from $I_{\rm ZLP}$
      and $I_{\rm inel}$ as well as their sum.
      %
      We indicate the two regions used for the model training ($I$ and $III$),
      while the trained model is interpolated to region $II$, 
      defined for $\Delta E_I \le \Delta E \le \Delta E_{II}$.}
    \label{fig:EELS_toy}
\end{figure}
%%%%%%%%%%%%%%%%%%%%%%%%%%%%%%%%%%%%%%%%%%%%%%%%%

The toy model of Fig.~\ref{fig:EELS_toy} is sufficiently general to draw
a number of useful considerations concerning the relation between $I_{\rm ZLP}$ and $I_{\rm inel}$
in realistic spectra:

\begin{itemize}

\item The ZLP intensity, $I_{\rm ZLP}(\Delta E)$, is a monotonically decreasing function
  and thus its derivative is always negative.

\item  The first local minimum of the total spectrum, $dI_{\rm EELS}/d\Delta E|_{\Delta E_{\rm min}}=0$, corresponds
  to a value of $\Delta E$ for which the contribution from the inelastic emissions is already
  significant.

\item The value of $\Delta E$ for which $I_{\rm inel}$ starts to contribute to the total spectrum
  corresponds to the position where the intensity derivatives in-sample and in-vacuum  start to differ.
  %
  We note that a direct comparison between the overal magnitude of the sample and vacuum ZLP
  spectra is in general not possible, as explained in Sect.~\ref{sec:eels}. 
\end{itemize}

These considerations suggest that when training the ML model on EEL spectra taken on samples,
the following categorisation should de adopted:

\begin{enumerate}

\item For small energy losses such that $\Delta E \le \Delta E_I$ (region $I$),
  the model training  proceeds in the same way as for the vacuum case
  via the minimisation of Eq.~(\ref{eq:chi2}).

\item  
  For $\Delta E \ge \Delta E_{II}$ (region $III$), we use instead Eq.~(\ref{eq:chi2modified})
  without the contribution from the input data, since for such values
  of $\Delta E$ one has that $I_{\rm inel}\gg I_{\rm ZLP}$.
  %
  In other words, we only fit on the pseudo data and therefore
  the only information that the region $III$ provides
  on the model is the one arising from the implementation
  of the constraint that $I_{\rm ZLP}(\Delta E\to \infty)\to 0$.

\item The EELS data in region $II$, defined by $\Delta E_I \le \Delta E \le \Delta E_{II}$,
  is excluded from the training dataset, given that in this region the contribution to $I_{\rm EEL}$
  coming from $I_{\rm inel}$ is already significant.
  %
  The model predictions in this regime are obtained from an interpolation
  of the predictions obtained in regions $I$ and $III$.

\end{enumerate}

This classification introduces two new hyper-parameters of our model, $\Delta E_I$ and
$\Delta E_{II}$, that need to be specified before the training.
%
They should satisfy $\Delta E_I \le \Delta E_{\rm min}$ and $\Delta E_{II} \ge \Delta E_{\rm min}$,
with $\Delta E_{\rm min}$ being the position of the first local minimum of $I_{\rm EEL}$.
%
Let's interpret this physically: $\Delta E_{\rm min}$ is the first local minimum, which means that 
the inelastic contributions are significantly present.
%
Training the data for higher than this value, we are sure that our training data includes
at least some amount of inelastic scatterings.
%
Therefore, it is certain that we need to cut the training data already at a lower energy loss,
so $\Delta E_I \le \Delta E_{\rm min}$.
%
As indicated by the toy spectra of Fig.~\ref{fig:EELS_toy}, a suitable value for $\Delta E_{I}$
would be somewhat above the onset of the inelastic contributions: this way we maximise
the amount of training data while ensuring that $I_{\rm EEL}$ is still dominated
by $I_{\rm ZLP}$.

We can use the derivatives of the spectra, $dI_{\rm EEL}/d\Delta E$, to select suitable minimum and
maximum values for $\Delta E_I$. 
%
Using first and second derivatives is an often-used feature extraction method to achieve a relatively 
high accuracy with a low computational complexity. 
%
In an ideal microscope the electron beam would be perfectly monochromatic, 
correspondingly the ZLP would appear as a delta function in an EEL spectrum \cite{Rafferty:2000}. 
%
In practice the ZLP has a finite width defining the energy resolution of the system. 
%
Concerning $\Delta E_I $, its minimum possible value is selected as the value where the derivate taken on the sample
data start to different significantly as compared to those spectra taken on vacuum.
%
This value is obtained by looking at the ratio of the derivatives of the spectra compared to the vacuum derivatives:
\be
R_{dI/d \Delta E} =  \left( \frac{dI_{\rm ZLP}/d\Delta E}{dI_{\rm EEL}/d\Delta E}\right). 
\ee
%
For low energy losses this ratio equals 1, which means that in the low-loss regime
the spectra recorded in vacuum and on specimens behave similar.
%
However, at some energy loss $\Delta E_{I,min}$ the
sample spectrum stops monotonically decreasing and the ratio deviates from 1. 
%
It is this point where the contributions from the sample start to change the shape of
the ZLP distribution and we can use this measure to mark the transition from 
regime I to regime II in Fig.~\ref{fig:EELS_toy}.
%
We know that the hyper-parameter $\Delta E_I$ should satisfy 
$\Delta E_{I,min} \le \Delta E_I \le \Delta E_{\rm min}$,
which corresponds to the region where $R_{dI/d \Delta E} \ne  1$ and $R_{dI/d \Delta E} \ge 0$.
%
The neural network will be trained on a range of different $\Delta E_I$ values within this interval,
and the optimal choice will be determined {\it a posteriori} from the results.
%
Also, generating results for different values of $\Delta E_I$ makes us able to 
cross-validate on the stability of our results regardless of the exact choice of this
hyperparameter.

Another important difference as compared to the training of the vacuum spectra is that each of the sample
spectra will have different values of $\Delta E_{\rm min}$ and thus of $\Delta E_I$. 
%
For this reason we calculate $\Delta E_{\rm min}$  for each of the sample spectra and we use the highest of these
as the maximum value for the hyper-parameter $\Delta E_I$. 
%
As we determine the best choice of $\Delta E_I$ after training for each of the spectra separately, 
we are sure to capture all suitable results and select the best value for each individual spectrum. \\

Concerning $\Delta E_{II}$, its minimum value should mark the region where $I_{\rm ZLP}(\Delta E\to \infty)\to 0$. 
%
It is the value where we start adding pseudo-data, so $\Delta E_{II}$ = $\Delta E_{\rm pd,min}$.
%
In order to implement this constraint, similar to the previous section (Eq.~(\ref{{eq:pdlocation}}))
we look at the ratio 
$\mathcal{R}_{\rm sig}(\Delta E_i)$ to determine the energy loss $\Delta E_{\rm pd}$ at 
which the contributions from the vacuum-recorded ZLP vanish. 
%
As a measure, we use the energy loss value where the ratio $\mathcal{R}_{\rm sig}(\Delta E_i)$ drops below 1,
as in this regime we would be fitting statistical noise.
%
We set the value of $\Delta E_{II}$ equal to this energy loss and add pseudo-data points for $\Delta E \ge \Delta E_{II}$.
%
Note that in this region the intensity of the ZLP is several orders of magnitude smaller than the intensity 
of the elastic emissions and therefore the exact choice of $\Delta E_{II}$ does not listen too closely.

Now that we have parametrised the 

\newpage

% Discussion of results in vacuum
% including closure test validation
\section{Results I. Vacuum spectra}
\label{sec:results_vacuum}

In this section we present the application of the previously presented strategy 
to the parametrisation of ZLP spectra acquired in vacuum.
%
Applying our model to this case has a two-fold motivation.
%
First of all, we aim to demonstrate that our model is flexible enough to effectively reproduce the
input EELS measurements for a range of variations of the operation parameters of the microscope.
%
Herefore, we modify the input parameters beyond those included in the training set
and see how the model is able to interpolate and extrapolate on its inputs.

Second, it will allow us to provide a calibration prediction
useful for the case of the in-sample measurements.
%
Such calibration is necessary since, as explained in Sect.~\ref{sec:training}, some of the model
hyper-parameters are determined by the comparison of the intensity derivatives
between spectra taken in vacuum and those in sample.

In this section, first of all we present the input dataset and motivate the choice
of training settings and model hyperparameters.
%
Then we validate the model training by assessing the fit quality.
%
Lastly, we study the dependence of the model predictions in its various input
variables, and study the dependence of the model uncertainties upon
the removal of a subset of the training dataset.

\subsection{Training initialization}

In Table~\ref{table:vacuumdata} we collect the main properties of the EELS spectra acquired in vacuum to train the neural
    network model.  For each set of spectra, we indicate the exposure time $t_{\rm exp}$, the beam energy
    $E_b$, the number of spectra $N_{\rm sp}$ recorded for these operation conditions, the number $N_{\rm dat}$ of
    bins in each spectrum, the range in electron energy loss $\Delta E$,
    and the average full width at half maximum (FWHM)
    evaluated over the $N_{\rm sp}$ spectra with the corresponding variance.
    %
    All the spectra listed on Table~\ref{table:vacuumdata}
    were acquired with a Titan TEM equipped with a Schottky field emitter.
    %
    We point out that since here
    we are interested in the low-loss region, $\Delta E_{\rm max}$ does not need
    to be too large, and in any case the large $\Delta E$ behaviour of the model is fixed
    by the constraint implemented by Eq.~(\ref{eq:chi2modified}).

%%%%%%%%%%%%%%%%%%%%%%%%%%%%%%%%%%%%%%%%%%%%%%%%%%%%%%%%%%%%%%%%%%%%%%%%%%%%%%%%%%%%%%%%%%%%%
%%%%%%%%%%%%%%%%%%%%%%%%%%%%%%%%%%%%%%%%%%%%%%%%%%%%%%%%%%%%%%%%%%%%%%%%%%%%%%%%%%%%%%%%%%%%%
\begin{table}[H]
  \begin{center}
            \renewcommand{\arraystretch}{1.50}
  \begin{tabular}{@{}ccccccccc}
\br
Set & $t_{\rm exp}$ {(}ms{)} & $E_{\rm b}$ {(}keV{)} & $N_{\rm sp}$ & $N_{\rm dat}$ & $\Delta E_{\rm min}$~(eV)  & $\Delta E_{\rm max}$~(eV)  & FWHM~(meV)  \\ 
\mr
1        & 100                 & 200                  & 15          & 2048               & -0.96              & 8.51     & $47\pm7 $         \\
2        & 100                 & 60                   & 7           & 2048               & -0.54              & 5.59    & 
$ 50 \pm 4$         \\
3        & 10                  & 200                  & 6          & 2048               & -0.75              & 5.18      & 
$ 26 \pm 3$         \\
4        & 10                  & 60                   & 6           & 2048               & -0.40              & 4.78       & 
$ 34\pm 2$         \\ 
\br
  \end{tabular}
    \end{center}
  \caption{\small Summary of the main properties of the EELS spectra acquired in vacuum to train the neural
    network model.  For each set of spectra, we indicate the exposure time $t_{\rm exp}$, the beam energy
    $E_b$, the number of spectra $N_{\rm sp}$ recorded for these operation conditions, the number $N_{\rm dat}$ of
    bins in each spectrum, the range in electron energy loss $\Delta E$,
    and the average FWHM evaluated over the $N_{\rm sp}$ spectra with the corresponding standard deviation
  }
   \label{table:vacuumdata}
\end{table}
%%%%%%%%%%%%%%%%%%%%%%%%%%%%%%%%%%%%%%%%%%%%%%%%%%%%%%%%%%%%%%%%%%%%%%%%%%%%%%%%%%%%%%%%%%%%%%%%%5
%%%%%%%%%%%%%%%%%%%%%%%%%%%%%%%%%%%%%%%%%%%%%%%%%%%%%%%%%%%%%%%%%%%%%%%%%%%%%%%%%%%%%%%%%%%%%

    The energy resolution of these spectra, quantified by the value of their FWHM, ranges
    from 26 meV to 50 meV depending on the specific operation conditions of the microscope,
    with an standard deviation between 2 and 7 meV.
    %
    The value of the FWHM varies only mildly with the value of the beam energy $E_b$
    but grows by around a factor two for spectra collected at larger exposure times $t_{\rm exp}$.
    %
    A total of almost $7\times 10^4$ independent measurements will be thus used for the ZLP model
    training on the vacuum spectra.
    %
    As will be highlighted in Sects.~\ref{eq:depdeltae} and~\ref{eq:depebeam}, one of the advantages of our ZLP model is that it can extrapolate its predictions
    to other operation conditions beyond those used for the training.

Following the strategy presented in Sect.~\ref{sec:methodology}, first of all we combine the $N_{\rm sp}$ spectra
corresponding to each of the four sets of operation conditions and determine the statistical uncertainty
associated to each energy loss bin by means of Eq.~(\ref{eq:sigmaiexp}).
%
The resulting set of training points can be observed in Fig.~\ref{fig:training_points_vacuum}.

%%%%%%%%%%%%%%%%%%%%%%%%%%%%%%%%%%%%%%%%%%%%%%%%%%%%%%%%%%%%%%%%%%%%%%%%%%%%%%%%%%%%%%%%%%%%%%%%%%%%%%%%%%%%
\begin{figure}[H]
    \centering
    \includegraphics[width=120mm]{plots/training_points_vacuum.pdf}
    \caption{Graphical representation of the four sets of training inputs
    that have been calculated as means and standard deviations over the
    $N_{\rm sp}$ four each set of operating conditions \{t$_{exp}$(ms), E$_b$(keV)\}.
      }
\label{fig:training_points_vacuum}
\end{figure}
%%%%%%%%%%%%%%%%%%%%%%%%%%%%%%%%%%%%%%%%%%%%%%%%%%%%%%%%%%%%%%%%%%%%%%%%%%%%%%%%%%%%%%%%%%%%%%%%%%%%%%%%%%%%%%%%%

For each of the training sets, we need to determine the value of $\Delta E_{\rm pd}^{\rm (min)}$
that defines the lower limit of the range for which we add the pseudo-data
that imposes the correct $\Delta E \to \infty$ limit of the model.
%
This value is determined
by inspecting the ratio between the central experimental value of the total
EELS intensity, $I_{{\rm EEL},i}^{(\rm exp)}$, and its corresponding
total uncertainty defined in Eq.~(\ref{eq:sigmaiexp}).

      
%%%%%%%%%%%%%%%%%%%%%%%%%%%%%%%%%%%%%%%%%%%%%%%%%%%%%%%%%%%%%%%%%%%%%%%%%%%%%%%%%%%%%%%%%%%%%%%%%%%%%%%%%%%%
\begin{figure}[H]
    \centering
    \includegraphics[width=120mm]{plots/intensity_to_error_ratio.pdf}
    \caption{The ratio between the central experimental value of the total
      EELS intensity, $I_{{\rm EEL},i}^{(\rm exp)}$, and the corresponding
      total uncertainty defined in Eq.~(\ref{eq:sigmaiexp}).
      %
      Results are shown for the four combinations of $t_{\rm exp}$
      and $E_{b}$ listed in Table~\ref{table:vacuumdata}.
      %
      The vertical dashed lines mark the values of $\Delta E$ for which
      this ratio becomes smaller than unity, which indicates when the input
      data starts to be dominated by the statistical noise.
      }
    \label{fig:intensityratio}
\end{figure}
%%%%%%%%%%%%%%%%%%%%%%%%%%%%%%%%%%%%%%%%%%%%%%%%%%%%%%%%%%%%%%%%%%%%%%%%%%%%%%%%%%%%%%%%%%%%%%%%%%%%%%%%%%%%%%%%%

Fig.~\ref{fig:intensityratio} displays this ratio
for the four combinations of $t_{\rm exp}$
and $E_{b}$ listed in Table~\ref{table:vacuumdata}.
%
The vertical dashed lines indicate the values of $\Delta E$ for which
this ratio becomes smaller than unity.
%
For larger $\Delta E$, the EELS spectra become
consistent with zero within uncertainties and can thus be discarded and replaced
by the pseudo-data constraints.
%
Thus the cross-over value of $\Delta E$  where the ratio satisfies $I/\sigma\simeq 1$
is a reasonable choice for $\Delta E_{\rm pd}^{\rm (min)}$
%
The total uncertainty of the pseudo-data points is chosen to be
\be
\sigma_j^{(\rm pd)} = \frac{1}{10}I_{{\rm EEL}}^{\rm (exp)}\lp \Delta E = \Delta E_{\rm pd}^{\rm (min)}\rp \,, \quad 
j= 1,\ldots,N_{\rm pd} \, .
\ee
The factor of 1/10 is found to be suitable to ensure that the constraint
is enforced without distorting
the training to the experimental data.
%
We note from Fig.~\ref{fig:intensityratio} that $\Delta E_{\rm pd}^{\rm (min)}$ will
depend on general on the operation conditions.
%
We find that
for our training samples $\Delta E_{\rm pd}^{\rm (min)} \simeq 200$ meV for $t_{\rm exp}=10$ ms
and $\simeq  900$ meV for 100 ms, roughly independent on the value of $E_b$.

The input experimental measurements listed in Table~\ref{table:vacuumdata} are used
to to generate a sample of $N_{\rm rep}=500$ Monte Carlo replicas
and train an individual neural network model to each of these replicas.
%
The end result of the procedure is a set of model replicas,
\be
\label{eq:modelreplicas}
I_{\rm ZLP}^{\rm (mod)(k)}(\Delta E, E_{b},t_{\rm exp}) \, , \quad k=1,\ldots,N_{\rm rep} \, ,
\ee
which can be used to provide a prediction for the intensity of the ZLP
for arbitrary values of $\Delta E$,  $E_{b}$, and $t_{\rm exp}$.
%
Eq~(\ref{eq:modelreplicas})
provides the sought-for representation of the probability density in the space of ZLP models.
%
For this sample of replicas one can evaluate
statistical estimators such as averages, variances, and correlations (as well
as higher moments) by means of
the usual expressions, for instance
\be
\label{eq:average}
\la I_{\rm ZLP}^{\rm (mod)}( \{z_1\}) \ra = \frac{1}{N_{\rm rep}}\sum_{k=1}^{N_{\rm rep}}
I_{\rm ZLP}^{\rm (mod)(k)}( \{z_1\}) \, ,
\ee
\be
\label{eq:standarddev}
\sigma_{I_{\rm ZLP}}^{\rm (mod)}( \{z_1\})  = \lp \frac{1}{N_{\rm rep}-1} \sum_{k=1}^{N_{\rm rep}}
\lp  I_{\rm ZLP}^{\rm (mod)(k)}  - \la I_{\rm ZLP}^{\rm (mod)}  \ra   \rp \rp^{1/2} \, ,
\ee
\be
\rho \lp \{z_1\},\{z_2\}\rp = \frac{ \la I_{\rm ZLP}^{\rm (mod)}( \{z_1\} ) I_{\rm ZLP}^{\rm (mod)}( \{z_2\} ) \ra
- \la I_{\rm ZLP}^{\rm (mod)}( \{z_1\} )\ra \la I_{\rm ZLP}^{\rm (mod)}( \{z_2\} ) \ra}{\sigma_{I_{\rm ZLP}}^{\rm (mod)}( \{z_1\} )\sigma_{I_{\rm ZLP}}^{\rm (mod)}( \{z_2\} )}
\ee
where as in the previous section $\{z_l\}$ denotes a possible set of input variables for the model.
We now discuss some of features of this ZLP vacuum model, here $\{z_l\}=\lp \Delta E, E_{b},t_{\rm exp}\rp$.

\subsection{Fit quality}
%
To begin with, we would like to quantify the overall fit quality of the model and demonstrate that it is flexible enough
to describe all the available input datasets.
%
In Table~\ref{table:chi2summary} we indicate the values of the final $\chi^2$ per data point,
    Eq.~(\ref{eq:chi2_final}), as well as the average values of the error Eq.~(\ref{eq:chi2})
    over the training and validation subsets, for each of the four sets of spectra listed in
    Table~\ref{table:vacuumdata} as well as for the total dataset.
    %
    We recall that for a satisfactory training one expects $\chi^2 \simeq 1$
    and $\la E_{\rm tr}\ra \simeq \la E_{\rm val}\ra \simeq 2 $~\cite{Forte:2002fg}.
    %
    From the results of this table we can see that these expectations are satisfied,
    both for the individual sets and for the total dataset.

    Then Fig.~\ref{fig:chi2_distributions} displays  the $\chi^2$  distributions
    evaluated for the training and validation sets
      of the $N_{\rm rep}=500$ replicas of the sample trained on the spectra
      listed in Table~\ref{table:vacuumdata}.
      %
      Note that the training/validation partition differs at random for each replica.
      %
      The $\chi^2_{\rm tr}$ distribution peaks at $\simeq .7$, indicating that a satisfactory model training
      has been achieved, however that the errors on our data points have been slightly overestimated.
      %
      We emphasize that the stopping criterion for the neural net training adopted here never considers
      the numerical values of the error function and determines proper learning entirely from
      the global minima of $E_{\rm val}^{(k)}$.
      %
      From Fig.~\ref{fig:chi2_distributions} we also observed that  $\chi^2_{\rm tr}$ distribution peaks at
      a slighter higher value, $\simeq 1$, and is broader that its corresponding training counterpart.
      %
      These results confirm both that a satisfactory model training that prevents overlearning
      has been achieved as well as an appropriate estimate of the statistical uncertainties
      affecting the original EEL spectra.
    
%%%%%%%%%%%%%%%%%%%%%%%%%%%%%%%%%%%%%%%%%%%%%%%%%%%%%%%%%%%%%%%%%%%%%%%%%%%%%%%%%%%%%%%%%%%%%%%%%%%%%%%%%%%%
\begin{figure}[H]
    \centering
    \includegraphics[width=0.49\textwidth]{plots/chi2_distributions.pdf}
    \includegraphics[width=0.49\textwidth]{plots/train_val_error_271.pdf}
    \caption{Left: The $\chi^2$ (per data point) distribution evaluated for the training and validation sets
      of the $N_{\rm rep}=500$ replicas of the sample trained on the spectra
      listed in Table~\ref{table:vacuumdata}. Right: the progress of the training and validation error over
      the course of the optimization. Number of replicas is taken small for clarity.
      The final $\chi^2$ is marked on the y-axis and shows how the distribution of training errors
      is more narrow and centered at a lower value compared to the validation errors. }
    \label{fig:chi2_distributions}
\end{figure}
%%%%%%%%%%%%%%%%%%%%%%%%%%%%%%%%%%%%%%%%%%%%%%%%%%%%%%%%%%%%%%%%%%%%%%%%%%%%%%%%%%%%%%%%%%%%%%%%%%%%%%%%%%%%%%%%%%%
%%%%%%%%%%%%%%%%%%

\begin{table}[H]
  \begin{center}
            \renewcommand{\arraystretch}{1.35}
  \begin{tabular}{@{}cccc}
\br
Set & $\chi^2$  &  $\la E_{\rm tr}\ra$   &  $\la E_{\rm val}\ra$ \\
\mr
1        &           0.998        &      1.702            &  1.970  \\
2        &           0.733        &     1.408            &  1.767  \\
3        &           0.697        &    1.391            &  1.800  \\
4        &           0.593        &    1.201            &  1.761  \\
\mr
Total    &           0.771        &    1.470            &  1.853  \\
\br
  \end{tabular}
    \end{center}
  \caption{\small \small The values of the final $\chi^2$ per data point,
    Eq.~(\ref{eq:chi2_final}), as well as the average values of the error Eq.~(\ref{eq:chi2})
    over the training and validation subsets, for each of the four sets of spectra listed in
    Table~\ref{table:vacuumdata} as well as for the total dataset.
  }
   \label{table:chi2summary}
\end{table}
%%%%%%%%%%%%%%%%%%%%%%%%%%%%%%%%%%%%%%%%%%%%%%%%%%%%%%%%%%%%%%%%%%%%%%%%%%%%%%%%%%%%%%%%%%%%%%%%%

From Table~\ref{table:chi2summary} we observe that the $\chi^2$ of the central fit is about half the size 
the error, which measures the quality of the fit for each MC replica. 
%
This is what we would expect a fit which correctly represents the fluctuations of the underlying set
of experimental data:
given that replicas fluctuate about the experimental measurements, these measurements themselves 
fluctuate about their "true" underlying values. 

%%%%%%%%%%%%%%%%%%%%%%%%%%%%%%%%%%%%%%%%%%%%%%%%%%%%%%%%%%%%%%%%%%%%%%%%%%%%%%%%%%%%%%%%%%%%%

\subsection{Dependence on the electron energy loss}
\label{sec:depdeltae}

Having demonstrated that our neural network model provides a satisfactory description
of all the input EEL spectra, we now present its predictions for specific
choices of the input parameters.
%
First of all, we investigate the dependence of the results as a function of the
electron energy loss $\Delta E$.
%
Fig.~\ref{fig:EELS_vacuum_DeltaE} displays the central value and 68\% confidence level uncertainty band
for the ZLP model as a function
of electron energy loss $\Delta E$
evaluated using Eqns.~(\ref{eq:average}) and~(\ref{eq:standarddev}).
%
We display results corresponding to 
three different values of $E_b$ and for both
$t_{\rm exp}=10$ ms (left)  and $t_{\rm exp}=100$ ms (right panel).
%
We emphasize that only beam energies of 60 and 200 keV were included in the dataset,
so the network is trained on these data.
%
It has never seen data with $E_b=120$ keV, and thus our prediction
in this case arises purely from the model interpolation.
%
It is interesting to note how both the overall normalisation and the shape of
the predicted ZLP depend on the specific operation conditions.

%%%%%%%%%%%%%%%%%%%%%%%%%%%%%%%%%%%%%%%%%%%%%%%%%%%%%%%%%%%%%%%%%%%%%%%%%%%%%%%%%%%%%%%%%%%%%%%%%%%%%%%%%%%%
\begin{figure}[H]
    \centering
    \includegraphics[width=150mm]{plots/deltaE_dependence_vacuum.pdf}
    \caption{Top: the central value and 68\% confidence level uncertainty band
      for the ZLP model as a function
      of electron energy loss $\Delta E$
      evaluated using Eqns.~(\ref{eq:average}) and~(\ref{eq:standarddev}).
      %
      We display results corresponding to 
      three different values of $E_b$  and for both
      $t_{\rm exp}=10$ ms (left)  and $t_{\rm exp}=100$ ms (right panel).
      %
      Note that no training data with $E_b=120$ keV has been used and thus our prediction
      in that case arises purely from the model interpolation.
      %
      Bottom: the corresponding relative uncertainty as a function of $\Delta E$
      for each of the three values of $E_b$.
      }
      \label{fig:EELS_vacuum_DeltaE}
\end{figure}
%%%%%%%%%%%%%%%%%%%%%%%%%%%%%%%%%%%%%%%%%%%%%%%%%%%%%%%%%%%%%%%%%%%%%%%%%%%%%%%%%%%%%%%%%%%%%%%%%%%%%%%%%%%%%%%%%%%

In the bottom panel of Fig.~\ref{fig:EELS_vacuum_DeltaE} we show
the corresponding relative uncertainty as a function of $\Delta E$
for each of the three values of $E_b$.
%
The relative uncertainty is calculated as the absolute error in each point divided
by the predicted intensity.
%
Recall that in this work we allow for non-Gaussian distributions and thus the central
value is the mean of the distribution while the error band in general will
be asymmetric.
%
Looking at the predictions for $t_{\rm exp}=10$ ms, we see how the model prediction
at $E_b=120$ keV typically exhibits larger uncertainties and is therefore less stable than the predictions
for the two values of $E_b$ for which we have training data.
%
In the case of $t_{\rm exp}=100$ ms instead, the model predictions exhibit very similar
uncertainties for the three values of $E_b$, which furthermore depend only mildly on $\Delta E$.
%
From these outcomes we can conclude that the network is well able to make predictions 
on inputs it has never seen before. 

It is interesting to assess how the model results change once a subset of the data points
is excluded from the fit.
%
In other words, we remove data points in a certain region 
[$\Delta E_{\rm cut}^{\rm (min)}$,$\Delta E_{\rm cut}^{\rm (max)}$]
and train the model on the resulting dataset. 
%
Afterwards, we let the model make predictions on the full energy range and see what the
effect of such a cut is. 
%
The reason to do this goes hand in hand with the fact that we will need this model later onwards
for the prediction on sample data. 
%
As explained in Sect.~\ref{sec:methodology} and as
illustrated in Fig.~\ref{fig:EELS_toy}, when training the model on sample spectra, a region
with $\Delta E_I \le \Delta E \le \Delta E_{II}$ will be removed from the training dataset to avoid the
contamination from the inelastic contributions.
%
Fig.~\ref{fig:EELS_vacuum_DeltaE_abs} displays
the predicted central value and uncertainty in the model predictions for $I_{\rm ZLP}(\Delta E)$
as a function of the energy loss for $E_b=200$ keV and $t_{\rm exp}=10$ ms (left)
and 100 ms (right panel).
%
We show results for three different sets of training: first of all, one without any cut
in the training dataset, and then for the case where the data points with $\Delta E \ge \Delta E_{\rm cut}$
are removed from the training dataset.
%
We consider two values of $\Delta E_{\rm cut}$, namely 50 meV and 100 meV, indicated
with vertical dash-dotted lines.
%
In both cases, data points are removed up until $\Delta E =$ 800 meV. The pseudo-data points 
that enforce $I_{\rm EEL}(\Delta E)\to 0$ are present
in all three cases in the region 800 meV $\le \Delta E \le 1 eV$. 

%%%%%%%%%%%%%%%%%%%%%%%%%%%%%%%%%%%%%%%%%%%%%%%%%%%%%%%%%%%%%%%%%%%%%%%%%%%%%%%%%%%%%%%%%%%%%%%%%%%%%%%%%%%%
\begin{figure}[H]
    \centering
    \includegraphics[width=0.49\textwidth]{plots/prediction_with_cut_10ms_absolute.pdf}
    \includegraphics[width=0.49\textwidth]{plots/prediction_with_cut_100ms_absolute.pdf}
    \caption{The central values and 68\% uncertainty bands 
    in the model predictions for $I_{\rm EEL}(\Delta E)$
      as a function of the energy loss for $E_b=200$ keV and $t_{\rm exp}=10$ ms (left)
      and 100 ms (right panel).
      %
      We show results for three different sets of trainings: without any cut
      in the training dataset, and for the case where the data points with $\Delta E \ge \Delta E_{\rm cut}$
      are removed from the training dataset for two different values
      of $\Delta E_{\rm cut}$.
      %
      Note that the same pseudo-data points that enforce $I_{\rm EEL}(\Delta E)\to 0$ are present
      in all three cases.
      }
      \label{fig:EELS_vacuum_DeltaE_abs}
\end{figure}
%%%%%%%%%%%%%%%%%%%%%%%%%%%%%%%%%%%%%%%%%%%%%%%%%%%%%%%%%%%%%%%%%%%%%%%%%%%%%%%%%%%%%%%%%%%%%%%%%%%%%%%%%%%%%%%%%%%


At first sight, the removal of data from the training dataset doesn't introduce large
uncertainties for the ZLP predictions: they are little noticed when the data is cut at 50meV,
and for 100 meV the predictions seem to identical to the full range training.
%
To emulate the effects of such cut, one needs to look at the {\it relative} uncertainty rather
than the absolute, that is, normalizing the uncertainty in each point to the predicted intensity.
%
The results can be observed in Fig.~\ref{fig:EELS_vacuum_cuts}. We show the relative uncertainty
in the model predictions using two different x and y scales, to visualize the results
as clearly as possible.

%%%%%%%%%%%%%%%%%%%%%%%%%%%%%%%%%%%%%%%%%%%%%%%%%%%%%%%%%%%%%%%%%%%%%%%%%%%%%%%%%%%%%%%%%%%%%%%%%%%%%%%%%%%%
\begin{figure}[H]
    \centering
    \includegraphics[width=0.49\textwidth]{plots/prediction_with_cut_10ms_zoomout.pdf}
    \includegraphics[width=0.49\textwidth]{plots/prediction_with_cut_100ms_zoomout.pdf}
    \includegraphics[width=0.49\textwidth]{plots/prediction_with_cut_10ms.pdf}
    \includegraphics[width=0.49\textwidth]{plots/prediction_with_cut_100ms.pdf}
    \caption{Top: The relative uncertainty in the model predictions for $I_{\rm EEL}(\Delta E)$
      as a function of the energy loss for $E_b=200$ keV and $t_{\rm exp}=10$ ms (left)
      and 100 ms (right panel).
      %
      Bottom: same results with different x and y scales for better visualization of the 
      differences in relative uncertainties. 
      }
      \label{fig:EELS_vacuum_cuts}
\end{figure}
%%%%%%%%%%%%%%%%%%%%%%%%%%%%%%%%%%%%%%%%%%%%%%%%%%%%%%%%%%%%%%%%%%%%%%%%%%%%%%%%%%%%%%%%%%%%%%%%%%%%%%%%%%%%%%%%%%%

From this comparison we can observe how the model predictions become significantly more unstable
once a subset of the training data is cut away, as expected due to the effect of the information
loss.
%
While for the cut $\Delta E_{\rm cut}=100$ meV the increase in model uncertainty is only moderate
as compared with the baseline fit where no cut is performed (since for this value of $\Delta E$
uncertainties are small to begin with), rather more dramatic effects are observed
for a value of the cut $\Delta E_{\rm cut}=50$ meV.
%
This comparison highlights how ideally we would like to keep as many data points
in the training set for the ZLP model, provided of course we can verify that the
possible contributions to the spectra related to inelastic scatterings from the
sample can be neglected.

\subsection{Dependence on beam energy and exposure time }
\label{sec:depebeam}

As indicated in Table~\ref{table:vacuumdata}, the training dataset contains
spectra taken at two values of the electron beam energy, $E_b=60$ keV and 200 keV.
%
The left panel of Fig.~\ref{fig:extrapolbeam} displays  model predictions for the FWHM of the zero loss peak
      (and its corresponding uncertainty) as a function of the beam energy $E_b$
      for two values of the exposure time, $t_{\rm exp}=10$ ms and 100 ms.
      %
      The vertical dashed lines indicate the values of $E_b$ for which training data is available.
      %
      This comparison illustrates how the model uncertainty vary in the data region
      (near $E_b=60$ keV and 200 keV), the interpolation region (for $E_b$ between 60 and 200 keV),
      and the extrapolation regions (for $E_b$ below 60 keV and above 200 keV).
      %
      For $t_{\rm exp}=100$ ms, we observe that the model interpolates reasonably well
      between the measured values of $E_b$ and then uncertainties increase
      markedly in the extrapolation region above $E_b=200$ keV.
      
%%%%%%%%%%%%%%%%%%%%%%%%%%%%%%%%%%%%%%%%%%%%%%%%%%%%%%%%%%%%%%%%%%%%%%%%%%%%%%%%%%%%%%%
\begin{figure}[H]
    \centering
    \includegraphics[width=0.49\textwidth]{plots/Ebeam_extrapolation.pdf}
    \includegraphics[width=0.49\textwidth]{plots/time_extrapolation.pdf}
    \caption{The model predictions for the FWHM of the zero loss peak
      (and its corresponding uncertainty) as a function of the beam energy $E_b$
      for two values of the exposure time (left panel)
      and as a function of $t_{\rm exp}$ for two values of $E_b$ (right panel).
      %
      The vertical dashed lines indicate the values of the
      corresponding parameter for which training data is available.
    }
\label{fig:extrapolbeam}
\end{figure}
%%%%%%%%%%%%%%%%%%%%%%%%%%%%%%%%%%%%%%%%%%%%%%%%%%%%%%%%%%%%%%%%%%%%%%%%%%%%%%%%%%%%%%%%%%%%

For this comparison one can observe that as expected the uncertainty in the  prediction for FWHM
of the ZLP is the smallest close to the values of $E_b$ for which one has training data.
%
The uncertainties increase but only in a moderate way in in the interpolation region, indicating that
the model can be reliable applied to predict the features of the ZLP other values of the electron
energy beam (assuming that all other operation conditions of the microscope are unchanged).
%
The errors increase rapidly in the extrapolation region, which is a characteristic feature of
these neural network models.
%
Indeed, as soon as the model departs from the data region there exist a very large
number of different functional form models for $I_{\rm ZLP}(\Delta E)$ that can describe equally well
the training dataset, and hence a blow up of the extrapolation uncertainties is generically expected.

The network was trained on data with exposure times of $10$ and $100$ ms,
so interpolation and extrapolation is possible. Similar to the predictions for varying beam energy, also for exposure time the uncertainties grow bigger as the value deviates more from the training inputs.
%
The right panel of Fig.~\ref{fig:extrapolbeam} displays a similar model
comparison as in the left one but now and as a function of $t_{\rm exp}$ for two values of $E_b$ (right panel).
%
We observe that the FWHM increases roughly in a linear manner with the exposure time, indicating
that lower values of $t_{\rm exp}$ allow for an improved spectral resolution.
%
Further, also in this case we find that the model uncertainties grow rapidly in the
extrapolation region beyond that covered for the training dataset.


\subsection{Stability and reliability checks}
In this section, we assess the stability of the results and the reliability of the 
error estimate regarding the neural network parametrisation.\\

To begin with, we study the depence on the architecture of the neural networks. 
As previously mentioned, we have chosen a redundant architecture as to assure that the network
is sufficiently large to parametrise the ZLP.
%
In order to check that this is indeed the case, we have both increased and decreased the
number of neurons, so that we can assess the stability of results for both cases.
%
Throughout this work, the default architecture is 3-10-15-5-1. 
The "big" architecture was chosen as a 3-15-25-10-1, increasing the number of free
parameters from 289 to 731. 
%
The "small" network corresponds to an architecture of 3-5-8-6-1, which introduces 129
free parameters to be optimised. 

We have followed a similar procedure as presented in Sect.~\ref{sec:depdeltae} to investigate
the dependence of the results as a function of the energy loss, corresponding to 
E$_b$ = 60, 120 and 200 keV. 
%
The central value and 68\% confidence level band for the predicted ZLP distributions
can be observed in Fig.~\ref{fig:EELS_vacuum_DeltaE_check} below, which can be directly
compared to Fig.~\ref{fig:EELS_vacuum_DeltaE}. 

\begin{figure}[H]
\centering
 \includegraphics[width=0.9\textwidth]{plots/Prediction_120keV_bignetwork.pdf}
 \includegraphics[width=0.9\textwidth]{plots/Prediction_120keV_smallnetwork.pdf}
 \caption{Similar to Fig.~\ref{fig:EELS_vacuum_DeltaE} but now
    for a network with twice (up) and half (down) the architecture. 
    We see that the when doubling the network, predictions are
    almost identical, exact that uncertainty bands are slightly smaller.
    However, decreasing the network by a factor of 2 results in predictions
    that are a bit off from our default network. 
    Note that the predictions made for E$_b$=120 keV are purely
    interpolation results; the training set contains only data for beam 
    energies of 60 and 200 keV.
    }
    \label{fig:EELS_vacuum_DeltaE_check}
\end{figure}
It can be observed that results are well
reproduced for both smaller and bigger graphs, . 
%
Particularly for the large network, both the central values and the uncertainty bands are 
in almost perfect correspondence with the results presented in Sect.~\ref{sec:depdeltae}. 
%
This confirms the fact that our default network is sufficiently large and flexible to parametrise
the ZLP: since increasing the size doesn't affect the obtained results, 
the number of optimizable parameters is much larger than the 
minimum required in order to reproduce the data.

This becomes more evident when we substantially decrease the size of the neural network.
Predictions on the training data (E$_b$=60 and 200 keV) are still consistent, however the uncertainty
bands have grown remarkably bigger, indicating that the neural network outcomes are less 
stable than before. 
%
When we look at the prediction for the interpolation (E$_b$=120 keV), the central 
values differ (however only moderately) from the ones obtained on bigger networks.
%
This difference indicates that the predicted outcomes indeed depend on the size of the network,
and simultaneously it confirms the sufficient redundancy of the architecture used as default. \\

Next, we check that the results are stable among different subsets of 100 out of the 
total of 500 Monte Carlo replicas. 
%
The results can be observed in Fig.~\ref{fig:EELS_vacuum_DeltaE_check2}.
%
We find remarkable stability: the two sets yield to almost identical representations,
both for the training data as for the interpolation regime. 
%
\begin{figure}[H]
\centering
 \includegraphics[width=0.9\textwidth]{plots/Prediction_120keV_100replicas_part1.pdf}
 \includegraphics[width=0.9\textwidth]{plots/Prediction_120keV_100replicas_part2.pdf}
 \caption{Similar to Fig.~\ref{fig:EELS_vacuum_DeltaE} but now for two subsets of 100 replicas
 instead of the full set of 500 replicas.
 }
\label{fig:EELS_vacuum_DeltaE_check2}
\end{figure}

A final check is to see how uncertainties change with the number of replicas. 
We have produced the same plot but now based on three different sizes of MC ensembles:
N$_{rep}$=25, 100 and 400. 
%
Again, we have created plots similar to the ones before, now using each time
a different size of replicas. 
%
We would expect that using a small subset introduces larger errors compared to 
large ensembles of replicas.
%
It is clear that for the smallest subset (N$_{rep}$=25), in the peaks the predictions are
already well produced, however uncertainties in the tails are significantly larger.
%
The difference between using either 100 or 400 replicas is hardly noticable, which
confirms our statement given in Sect.~\ref{sec:uncertaintypropagation} that 500
replicas is (more than) enough to obtain a faithful representation.

\begin{figure}[H]
\centering
 \includegraphics[width=0.7\textwidth]{plots/Prediction_120keV_different_nrep.pdf}
 \caption{Predictions for three different number of replicas. 
 See how uncertainties change..}
\label{fig:EELS_vacuum_DeltaE_check3}
\end{figure}




\newpage

% Discussion of the results in sample
\section{Results II. Sample spectra}
\label{sec:results_sample}

Following the discussion of the vacuum ZLP analysis, we now
present the application of our machine learning strategy to parametrise the ZLP
arising in spectra recorded on specimens, specifically for
EELS measurements acquired in different regions
of the WS$_2$ nanoflowers presented in Sect.~\ref{sec:tmd}.
%
The resulting ZLP parametrisation will be applied to isolate the inelastic
contribution in each spectrum.
%
We will use these subtracted spectra first to determine the bandgap type and energy 
value from the behaviour of the onset region and second to study features
in the very-low-loss region.

We start this section by presenting the training dataset, which consists of two groups of EEL spectra recorded
in thick and thin regions of the WS$_2$  nanoflowers respectively.
%
Then we discuss the subtraction procedure, the choice of hyper-parameters, and the error propagation
to the physical predictions.
%
The resulting subtracted spectra provide the information
required to extract the value and type of the bandgap
and to characterise excitonic transitions for different regions of these polytypic WS$_2$ nanostructures.


\subsection{Training dataset}
%

Low-magnification TEM images and the corresponding
spectral images of two representative regions of
the WS$_2$ nanoflowers, denoted as sample A and B  respectively, are displayed in Fig.~\ref{fig:ws2positions}.
%
These spectral images have been recorded in the regions marked by a green square
in the associated TEM images, and contain an individual EEL spectrum in each pixel.
%
We indicate the specific locations where
EEL spectra have been recorded, including the in-vacuum measurements acquired
for calibration purposes.
%
Note that in sample B  the differences in contrast are related to the material
thickness, with higher contrast corresponding to thinner regions.

These two samples are characterised by rather different structural morphologies.
%
While sample A is a relatively thick region of WS$_2$, sample B corresponds to a region composed 
of thin petals, which is only one or a few monolayers thick. 
%
In other words, sample A is composed by bulk WS$_2$ while in sample B some specific regions
are much thinner, down to the a few or single monolayer level.
%
This thickness information has been be determined
by means of the {\tt Digital~Micrograph} software.

%%%%%%%%%%%%%%%%%%%%%%%%%%%%%%%%%%%%%%%%%%%%%%%%%%%%%%%%%%%%%%%%%%%%%%%
\begin{figure}[H]
\begin{centering}
  \includegraphics[width=0.87\linewidth]{plots/Spectra_location2.pdf}
  \caption{Low-magnification TEM images (left) and the corresponding
    spectral images (right panels) of two different regions of
    the WS$_2$ nanoflowers, denoted as sample A (upper) and sample B (lower panels) respectively.
    %
    The spectral images have been recorded in the regions marked by a green square
    in the associated TEM images, and contain an individual EEL spectrum in each pixel.
    %
    We indicate the locations where representative
    EEL spectra have been selected. 
    %
    In the left panel of sample B, the difference in contrast is correlated to the material
    thickness, with higher contrast indicating thinner regions of the nanostructure.
    %
    The morphological differences between the two samples are discussed in the text.
  }
\label{fig:ws2positions}
\end{centering}
\end{figure}
%%%%%%%%%%%%%%%%%%%%%%%%%%%%%%%%%%%%%%%%%%%%%%%%%%%%%%%%%%%%%%%%%%%%%%%%%%

One of the main goals of this study is demonstrating that our ZLP-subtraction method leads to
a satisfactory performance for spectra recorded with different microscopes and operating conditions.
%
With this motivation, the EELS measurements acquired on specimens A and B have
been obtained varying both the microscopes and their settings.
%
The TEM and EELS measurements acquired in specimen A  are based on a JEOL 2100F
microscope with a cold field-emission gun and equipped with an aberration corrector,
operated at 60 kV. A Gatan GIF Quantum energy filter was used for
the EELS analysis.
%
The corresponding measurements on specimen B were recorded instead
using a JEM ARM200F monochromated microscope operated at 60 kV, equipped with a GIF quantum ERS filter.
%
See the Methods section at the end of this work for more details.\\

In Table~\ref{table:sampledata} we collect the most relevant properties of the spectra collected
in the locations indicated in Fig.~\ref{fig:ws2positions} using the same format as
in Table~\ref{table:vacuumdata}.
%
Note that since the spectra from samples A and B
have been acquired with different microscopes, features of the ZLP
such as the FWHM are expected to be different.
%
From this table one can observe how the ZLP for the spectra acquired on sample A exhibit
a FWHM about five times larger as compared to those of sample B.
%
This difference in resolution can be understood from the fact that the EELS spectra from sample B, unlike those
from sample A, were recorded with a TEM equipped with a monochromator.

%%%%%%%%%%%%%%%%%%%%%%%%%%%%%%%%%%%%%%%%%%%%%%%%%%%%%%%%%%%%%%%%%%%%%%%%%%%%%%%%%%%%%%%%%%%%%
%%%%%%%%%%%%%%%%%%%%%%%%%%%%%%%%%%%%%%%%%%%%%%%%%%%%%%%%%%%%%%%%%%%%%%%%%%%%%%%%%%%%%%%%%%%%%
\begin{table}[H]
  \begin{center}
            \renewcommand{\arraystretch}{1.50}
  \begin{tabular}{@{}ccccccccc}
\br
Set & $t_{\rm exp}$ {(}ms{)} & $E_{\rm b}$ {(}keV{)} & $N_{\rm sp}$ & $N_{\rm dat}$ & $\Delta E_{\rm min}$~(eV)  & $\Delta E_{\rm max}$~(eV)  & FWHM~(meV)  \\ 
\mr
A        &       1       &        60         &   6      &    1918    &     -4.1       & 45.5 & $ 470\pm 10$  \\
B        &       190       &        60       &   10     &    2000    &     -0.9        & 9.1   & $ 87 \pm 5$ \\
\br
\end{tabular}
\end{center}
\caption{Same as Table~\ref{table:vacuumdata} now for the EEL spectra taken on specimens A and B.
    %
Note that the location on the WS$_2$ nanoflowers where each spectra has been recorded
was indicated in Fig.~\ref{fig:ws2positions}.}
\label{table:sampledata}
\end{table}
%%%%%%%%%%%%%%%%%%%%%%%%%%%%%%%%%%%%%%%%%%%%%%%%%%%%%%%%%%%%%%%%%%%%%%%%%%%%%%%%%%%%%%%%%%%%%%%%%5
%%%%%%%%%%%%%%%%%%%%%%%%%%%%%%%%%%%%%%%%%%%%%%%%%%%%%%%%%%%%%%%%%%%%%%%%%%%%%%%%%%%%%%%%%%%%%

In the following we will present the results for spectra that are representative
for each of the two samples.
%
The full set of spectra is available together with {\tt EELSfitter},
the code used to produce the results of this analysis,
whose installation
and usage instructions are presented in Appendix~\ref{sec:installation}.

\subsection{Subtraction procedure}

Following the strategy presented in Sect.~\ref{sec:methodology}, again we combine
the N$_{sp}$ spectra recorded over each sample and we determine the 
experimental central values and uncertainties for the training points.
%
Next, we need to determine the choice for hyperparameters $\Delta E_I$ and $\Delta E_{II}$,
as we only want to train on data that is different from zero within uncertainties.
%
As explained in Sect.~\ref{sec:training}, we need the location of the first local minimum
on all the spectra to set a bound for $\Delta E_I$ and $\Delta E_{II}$. 
%
The representative plots can be observed in Fig.~\ref{fig:derivatives} and here it becomes
clear how much information is contained within the intensity derivatives of the spectra.
%
One can observe how the derivatives recorded in vacuum (corresponding to locations
\#1, \#2, and \#3 for both samples) are monotonically decreasing and slowly converge towards zero, whereas
the derivatives recorded on different positions over a specimen behave very similar, capturing fluctuations
that are typical for the corresponding sample.

%%%%%%%%%%%%%%%%%%%%%%%%%%%%%%%%%%%%%%%%%%%%%%%%%%%%%%%%%%%%%%%%%%%%%%%
\begin{figure}[H]
\begin{centering}
  \includegraphics[width=0.49\linewidth]{plots/Derivatives_sample_A.pdf}
  \includegraphics[width=0.49\linewidth]{plots/Derivatives_sample_B.pdf}
  \caption{Plot of the first derivatives of the intensity profile over the energy loss, 
  corresponding to the spectra
  recorded over Sample A (left) and B (right) at the locations
  indicated in Fig.~\ref{fig:ws2positions}.
  %
  The derivatives have been normalized to the corresponding intensity in order to 
  capture the relative size of the fluctuations.
  }
\label{fig:derivatives}
\end{centering}
\end{figure}
%%%%%%%%%%%%%%%%%%%%%%%%%%%%%%%%%%%%%%%%%%%%%%%%%%%%%%%%%%%%%%%%%%%%%%%%%%

\begin{table}[H]
  \begin{center}
            \renewcommand{\arraystretch}{1.50}
  \begin{tabular}{@{}ccccccccc}
\br
Set & $\Delta E|_{\rm min}$~(eV)  &  $\Delta E_{\rm I}$~(eV)  &  $\Delta E_{\rm II}$~(eV)   \\
\mr
A        &    $2.70\pm0.06$               &          1.8        &      12       \\
B        &    $1.80\pm0.04$               &          1.4        &      6        \\
\br
  \end{tabular}
  \end{center}
  \caption{The mean value and uncertainty of the first local minima, $\Delta E|_{\rm min}$,
    averaged over the spectra corresponding to samples A and B from
    Fig.~\ref{fig:ws2positions}.
    %
    We also indicate
    the corresponding values of the hyper-parameters
    $\Delta E_{\rm I}$ and $\Delta E_{\rm II}$ defined in Fig.~\ref{fig:EELS_toy} used for the training
    of the neural network model.
    %
  }
\label{table:sampledata_summary}
\end{table}
%%%%%%%%%%%%%%%%%%%%%%%%%%%%%%%%%%%%%%%%%%%%%%%%%%%%%%%%%%%%%%%%%%%%%%%%%%%%%%%%%%%%%%%%%%%%%%%%%

In Table~\ref{table:sampledata_summary} we have collected the mean value 
and uncertainty of the first local minimum, $\Delta E|_{\rm min}$,
averaged over the spectra corresponding to samples A and B from
Fig.~\ref{fig:ws2positions}.
%
From the uncertainties in $\Delta E|_{\rm min}$, we see that the
location of the first minimum is relatively stable
among all the spectra belonging to a given set.
%
This indicates that the onset of the inelastic contributions $I_{\rm inel}$ does
not change significantly as we move to different regions of the sample.
%
We also indicate
the corresponding values of the hyper-parameters
$\Delta E_{\rm I}$ and $\Delta E_{\rm II}$ defined in Fig.~\ref{fig:EELS_toy}.
%
Recall that only
the data points with $\Delta E \le \Delta E_{\rm I}$ are used for training
the neural network model.
%
For $\Delta E \ge \Delta E_{\rm II}$ instead, the training set includes only the pseudo-data
that implements the $I_{\rm ZLP}(\Delta E)\to 0$ constraint.
%
The values for $\Delta E_{\rm II}$ were determined from the vacuum recorded spectra
following the same procedure as explained 
in Sect.~\ref{sec:results_vacuum} and a plot similar to Fig.~\ref{fig:intensityratio} 
can be observed in  Fig.~\ref{fig:pdlocsample}, where we show the ratio between the central
experimental value of the vacuum EEL intensity and their corresponding uncertainty.
%
%%%%%%%%%%%%%%%%%%%%%%%%%%%%%%%%%%%%%%%%%%%%%%%%%%%%%%%%%%%%%%%%%%%%%%%
\begin{figure}[H]
\begin{centering}
  \includegraphics[width=0.49\linewidth]{plots/delta2_sampleA.pdf}
  \includegraphics[width=0.49\linewidth]{plots/delta2_sampleB.pdf}
  \caption{The ratio between the central experimental value of the total EELS intensity 
  of the vacuum recorded peaks, I$^{(exp)}_{vac}$,
  and their corresponding uncertainty defined in Eq.~\ref{eq:pdlocation}.
  %
  Results are shown for the mean of the spectra recorded over Sample A (left) and B (right). 
  %
  For both samples, spectrum positions \#1, \#2 and \#3 in Fig.~\ref{fig:ws2positions}
  were used to construct the vacuum central values.
  %
  The vertical dashed lines mark the values of $\Delta E$ for which this ratio becomes smaller 
  than unity, which indicates when the input data starts to be dominated by the statistical noise. 
  }
\label{fig:pdlocsample}
\end{centering}
\end{figure}
%%%%%%%%%%%%%%%%%%%%%%%%%%%%%%%%%%%%%%%%%%%%%%%%%%%%%%%%%%%%%%%%%%%%%%%%%%

We note that the values of $\Delta E_{\rm II}$ for this part are significantly higher than
the ones found in Fig.~\ref{fig:intensityratio}. This could be ascribed to the fact that 
the vacuum spectra from sample A and B are recorded in proximity to a sample, and therefore
effects from the sample are still present, although at a reduced rate.\\

The model training is performed for a range of $\Delta E_{\rm I}$ values
subject to the condition that $\Delta E_{\rm I} \le \Delta E_{\rm min}$.
%
The optimal values of $\Delta E_{\rm I}$ listed in Table~\ref{table:sampledata_summary} are
determined as follows.
%
We evaluate the ratio
between the derivative of the intensity distribution acquired on the specimen over the
same quantity recorded in vacuum,
\be
\label{eq:rder}
\mathcal{R}^{(j)}_{\rm der}(\Delta E) \equiv
\la
\frac{
  dI_{\rm EEL}^{({\rm exp})(j)}(\Delta E)/ d\Delta E
}{
  dI_{\rm EEL}^{({\rm exp})(j')}(\Delta E) /d\Delta E
} \ra_{N_{\rm sp}' } \, ,
\ee
where $j'$ labels one of the $N_{\rm sp}'$ vacuum spectra and the average is taken
over all available values of $j'$.
%
This ratio allows to identify a suitable value of $\Delta E_{I}$ by establishing
for which energy losses the shape (rather than the absolute value) of the intensity distributions 
recorded on the specimen starts to differ significantly from their vacuum counterparts.
%
A sensible choice of $\Delta E_{\rm I}$ could for instance be given by
$\mathcal{R}_{\rm der}(\Delta E_{\rm I}) \simeq 0.9$, for which derivatives differ
at the 10\% level.
%
Note also that the value of the energy loss satisfying
$\mathcal{R}_{\rm der}(\Delta E)=0$ in Eq.~(\ref{eq:rder}) corresponds to the position of the first
local minimum of the spectrum.

For both Sample A and B, the ratios calculated by means of Eq.~(\ref{eq:rder}) 
can be observed in Fig.~\ref{fig:rder} for two representative spectrum locations: 
we have used loc \#4 in both cases. Similar results are obtained for different
positions.

%%%%%%%%%%%%%%%%%%%%%%%%%%%%%%%%%%%%%%%%%%%%%%%%%%%%%%%%%%%%%%%%%%%%%%%
\begin{figure}[H]
\begin{centering}
  \includegraphics[width=0.49\linewidth]{plots/Derivatives_ratio_A.pdf}
  \includegraphics[width=0.49\linewidth]{plots/Derivatives_ratio_B.pdf}
  \caption{Plot of the ratio
between the derivative of the intensity distribution 
acquired on the specimen over the
same quantity recorded in vacuum,
  for spectrum position \#4 in Sample A (left) and \#4 in Sample B (right),
  indicated in Fig.~\ref{fig:ws2positions}.}
\label{fig:rder}
\end{centering}
\end{figure}
%%%%%%%%%%%%%%%%%%%%%%%%%%%%%%%%%%%%%%%%%%%%%%%%%%%%%%%%%%%%%%%%%%%%%%%%%%

In specimen A it is straightforward to find that $\mathcal{R}_{\rm der}(\Delta E_I) \simeq 0.9$
at an energy loss of 1.8 eV. 
%
For specimen B,  $\mathcal{R}_{\rm der} \simeq 0.9$ at $\Delta E$=1.4 eV.
%
In this case however, the ratio first starts to increase before it decreases again. 
%
This happens consistently for all three vacuum spectra: the three individual ratios can hardly
be distinguished. 
%
It should be noted that the derivatives also differ at the 10\% level when 
$\mathcal{R}_{\rm der} \simeq 1.1$, which corresponds to $\Delta E$=1 eV.
%
Still, we use the point where $\mathcal{R}_{\rm der} \simeq 0.9$
for the determination of $\Delta E_{\rm I}$.
%
The reason to favor this choice over $\mathcal{R}_{\rm der}(\Delta E_I) \simeq 1.1$ is from 
the physical interpretation: after this point,
the derivative of the intensity distributions acquired on samples starts 
to approximate zero much faster than the derivatives in vacuum.
%
When the effect of the sample kicks in, the intensity profile should flatten or increase rather than
continue decreasing, which means the derivatives ratio should be {\it smaller} than unity. 
%
For this reason we choose $\Delta E_I$=1.4 eV as hyper-parameter for specimen B. 
Note that the validity of this choice will be checked later on.\\

Now that we have verified our choices of $\Delta E_I$ and $\Delta E_{II}$ for both samples,
we can move to the results of the training sessions. 
%
The end result of the neural network training is
 a set of $N_{\rm rep}=500$ replicas
  parametrising the ZLP, $I_{\rm ZLP}^{({\rm mod})(k)}(\Delta E)$.
 %
 Taking into account that we have $N_{\rm sp}$ individual spectra in each sample,  the ZLP
 subtraction is performed individually
 for each Monte Carlo replica,
 \be
 \label{eq:subtractedModelPrediction}
 I_{\rm inel}^{({\rm exp})(j,k)}(\Delta E) \equiv I_{\rm EELS}^{({\rm exp})(j)}(\Delta E) - I_{\rm ZLP}^{({\rm mod})(k)}(\Delta E)\, ,
 \quad \forall~N_{\rm rep} \, ,\quad j=1,\ldots,N_{\rm sp} \, ,
 \ee
 from which statistical estimators can be evaluated as usual.
 %
 For instance, the mean value for our model prediction for the $j$-th spectrum
 can be evaluated by averaging over the set of replicas,
 \be
 \la  I_{\rm inel}^{({\rm exp})({\rm (exp)}j)}\ra (\Delta E)
 = \frac{1}{N_{\rm rep}} \sum_{k=1}^{N_{\rm rep}}  I_{\rm inel}^{({\rm mod})(j,k)}(\Delta E) \, ,
 \quad j=1,\ldots,N_{\rm sp} \, ,
 \ee
 and likewise for the corresponding uncertainties or correlations.
%
 For large values of $\Delta E$
 the model prediction reduces to the original spectra, since in that region
 the ZLP contribution vanishes,
 \be
 I_{\rm inel}^{({\rm exp})(j,k)}(\Delta E \gg \Delta E_{\rm I}) \to  I_{\rm EEL}^{{\rm (exp)}(j)}(\Delta E) \, ,\quad
 \forall~j,k \, .
 \ee
 
 For very small values of the energy loss, the contribution to the total
 spectra from inelastic scatterings is negligible
 and thus the subtracted model prediction Eq.~(\ref{eq:subtractedModelPrediction}) should
 be zero. 
 %
 However, this will not be the case in practice since the neural-net model is trained on
 the $N_{\rm sp}$ ensemble of spectra, rather that just on individual ones, and thus the expected
 $\Delta E \to 0$ behaviour will only be achieved within uncertainties rather than at the level of
 central values.
 %
 Therefore, in the zero-loss regime the subtracted spectrum will not vanish completely.
 %
 To achieve the desired $\Delta E \to 0$ limit, we apply a matching procedure
 as follows.
 %
 We introduce another hyper-parameter, $\Delta E_0 < \Delta E_1$, such that
 one has for the $k$-th ZLP replica associated to the $j$-th spectrum the following
 behaviour:
 \bea
 \nonumber
 I_{\rm ZLP}^{({\rm mod})(j,k)}(\Delta E) &=& I_{\rm EELS}^{({\rm exp})(j)}(\Delta E) \, ,\quad \Delta E < \Delta E_0  \, ,\\
 I_{\rm ZLP}^{({\rm mod})(j,k)}(\Delta E) &=& I_{\rm EELS}^{{\rm (exp)}(j)} + \lp \xi_1^{(n_l)(k)}(\Delta E) -
 I_{\rm EELS}^{{\rm (exp)}(j)}(\Delta E)\rp  \times \mathcal{F} \, , \nonumber \quad 
 \Delta E_0 < \Delta E \le \Delta E_1 \, ,\\
 &&\mathcal{F} = \exp\lp -\frac{\lp \Delta E - \Delta E_1 \rp^2 }{\lp \Delta E_0 - \Delta E_1 \rp^2 \delta^2} \rp  \, , \label{eq:matching} \\
 I_{\rm ZLP}^{({\rm mod})(j,k)}(\Delta E) &=& \xi_1^{(n_l)(k)}(\Delta E) \, , \quad \Delta E > \Delta E_1 \nonumber \, .
 \eea
In Eq.~(\ref{eq:matching}), $\xi_1^{(n_l)(k)}$ indicates the output of the $k$-th neural network that parametrises
 the ZLP and $\delta$ is a dimensionless tunable parameter.
 %
This matching procedure might look complex at first sight, however it states just the following:
\begin{itemize}
\item For $\Delta E < \Delta E_0$, the modeled ZLP is exactly the same as the original spectrum.
\item Between $\Delta E_0$ and $\Delta E_1$, the transition sets in to the ZLP model prediction. $\mathcal{F}(\Delta E)$ represents a matching factor
 that ensures that the ZLP model prediction smoothly interpolates
 between $\Delta E=\Delta E_0$ (where $\mathcal{F}\ll 1$ and the original spectrum should
 be reproduced) and $\Delta E=\Delta E_1$
 (where $\mathcal{F}=1$ leaving the neural network output unaffected).
\item At energy loss higher dan $\Delta E_I$, the modeled ZLP is exactly the network prediction.
\end{itemize}

 %
 Here we adopt $\Delta E_0 = \Delta E_1 -0.5\,{\rm eV}$,  having verified
 that results are fairly independent of this choice.
 %
 Taking into account the matching procedure, we can slightly modify Eq.~(\ref{eq:subtractedModelPrediction})
 to 
 \be
 \label{eq:subtractedModelPrediction2}
 I_{\rm inel}^{({\rm mod})(j,k)}(\Delta E) \equiv I_{\rm EELS}^{({\rm exp})(j)}(\Delta E) - I_{\rm ZLP}^{({\rm mod})(j,k)}(\Delta E)\, ,
 \quad \forall~N_{\rm rep} \, ,\quad j=1,\ldots,N_{\rm sp} \, .
 \ee

 The ensemble of ZLP-subtracted spectra $\{I_{\rm inel}^{({\rm mod})(j,k)} \} $
 can then be used to estimate the bandgap of the specimen in the region where
 they were acquired.
 %
 Different approaches  have been put forward to estimate the value of the bandgap from 
subtracted EEL spectra, \textit{e.g.} by means of the inflection point of the rising intensity or
a linear fit to the maximum positive slope~\cite{Schamm:2003}.
%
Here we will adopt the approach of~\cite{Rafferty:2000} where the behaviour
of $I_{\rm inel}(\Delta E)$ in the onset region is modeled as
\begin{equation}
  \label{eq:I1}
    I_{\rm inel}(\Delta E) \simeq  A \lp \Delta E-E_{\rm BG} \rp^{b} \, , \quad \Delta E \ge E_{\rm BG} \, ,
\end{equation}
and vanishes for $E < E_{\rm BG}$, where both the bandgap value
$E_{\rm BG}$ as well as the parameters $A$ and $b$ are extracted from the fit.
%
The exponent $b$ is expected to be $b\simeq 1/2~(3/2)$ for a semiconductor material characterised
by a direct~(indirect) bandgap.
 %
 For each of the $N_{\rm sp}$ spectra and the $N_{\rm rep}$ replicas
 we fit to Eq.~(\ref{eq:subtractedModelPrediction2}) the model Eq.~(\ref{eq:I1})
 within a range taken to be
 $\lc \Delta E_{\rm I} - 0.5~{\rm eV}, \Delta E_{\rm I} + 0.7~{\rm eV}\rc$.
 %
 One ends up with $N_{\rm rep}$ values for
 the bandgap energy and fit exponent for each spectra,
 \be
 \Big \{ E_{\rm BG}^{(j,k)}, b^{(j,k)} \Big\} \, , \quad k=1,\ldots,N_{\rm rep} \, ,
 \quad j=1,\ldots,N_{\rm sp} \, ,
 \ee
 from which again one can readily evaluate their statistical estimators.
 %
 In the following, we will display the median and the 68\% confidence level intervals
 for these parameters to account for the fact that their distribution will be in general non-Gaussian.

 \subsection{Bandgap analysis of sample A}

We present first the results of the bandgap analysis of sample A,
taking location \#4 and \#5 in Fig.~\ref{fig:ws2positions} as representative spectra; 
compatible results are found for the rest of locations.
%
As mentioned above, this region is characterised by a sizable thickness where
WS$_2$ is expected to behave as a bulk material.
%
Fig.~\ref{fig:sp14_subtracted_spectrum} displays the original
and subtracted EEL spectra
together with the predictions of the ZLP model, where
the bands indicate the 68\% confidence level uncertainties and the central value
is the median of the distribution.
%
The inset shows the result of the polynomial fits using Eq.~(\ref{eq:I1}) to the subtracted spectrum
together with the corresponding uncertainty bands.

%%%%%%%%%%%%%%%%%%%%%%%%%%%%%%%%%%%%%%%%%%%%%%%%%%%%%%%%%%%%%%%%%%%%%%%
\begin{figure}[H]
\begin{centering}
  \includegraphics[width=0.49\linewidth]{plots/SubtractedEELS_plot_sp14.pdf}
  \includegraphics[width=0.49\linewidth]{plots/SubtractedEELS_plot_sp15.pdf}
   \caption{The original
     and subtracted EEL spectrum corresponding to location \#4 (left) and \#5 (right)
     of sample A in Fig.~\ref{fig:ws2positions},
     together with the predictions of the ZLP model, where
     the bands indicate the 68\% confidence level uncertainties.
     %
     The inset displays the result of fitting Eq.~(\ref{eq:I1}) to the onset
     region of the subtracted spectrum.
  }
\label{fig:sp14_subtracted_spectrum}
\end{centering}
\end{figure}
%%%%%%%%%%%%%%%%%%%%%%%%%%%%%%%%%%%%%%%%%%%%%%%%%%%%%%%%%%%%%%%%%%%%%%%%%%

One can observe how the ZLP model uncertainties are small at low $\Delta E$
(due to the matching condition) and large $\Delta E$ (where the ZLP vanishes),
but become significant in the intermediate region where the contributions
from $I_{\rm ZLP}$ and $I_{\rm inel}$ become comparable.
%
It is worth emphasizing that these (unavoidable) uncertainties are neglected in most
ZLP subtraction methods, and this is therefore the power of this method.
%
The validity of our choice for the hyperparameter $\Delta E_{\rm I}$ (Table~\ref{table:sampledata_summary})
can be verified {\it a posteriori} by evaluating the ratio
\be
\mathcal{R}^{(j)}_{\rm abs}\lp \Delta E_{\rm I}\rp \equiv 
\la I_{\rm ZLP}^{({\rm mod})(j)}\ra_{\rm rep} \Big/I_{\rm EEL}^{({\rm exp})(j)} \Big|_{\Delta E = \Delta E_{\rm I}} \, ,
\ee
which in this case turns out to be $\mathcal{R}_{\rm abs} = 0.98$.
%
It is important to verify that $\mathcal{R}_{\rm abs}\lp \Delta E_1\rp$ is not too far from unity,
indicating that the training dataset has not been contaminated by the inelastic contributions.

By requiring that $\mathcal{R}^{(j)}_{\rm der}(\Delta E_{\rm I})\simeq 0.9$ we obtain
the value $\Delta E_{\rm I}=1.8$ eV, which is used as baseline for the analysis.
%
It should be noted that this choice is not unique, for example requiring
$\mathcal{R}^{(j)}_{\rm der}(\Delta E_{\rm I})\simeq 0.8$ instead would have led
to $\Delta E_{\rm I}=2.0$ eV.
%
It is therefore important to asses the stability of our results when the hyper-parameter $\Delta E_{\rm I}$
is varied around its optimal value.

With this motivation, we have performed the training over the EEL spectra 
for a range of $\Delta E_{\rm I}$ values to assess the stability of our
results.
%
In Fig.~\ref{fig:bvalues_sampleA} we display the
values of the exponent $b$
and the bandgap energy $E_{\rm BG}$ 
obtained from spectrum \#4 (left panel in Fig.~\ref{fig:sp14_subtracted_spectrum})
for variations of $\Delta E_{\rm I}$ around its optimal value
(1.8 eV, indicated by the horizontal dashed line) by an amount
of $\pm 0.2$ eV.
%
The central value and the error band for each value of $\Delta E_I$ is evaluated
as the median and the 68\% CL interval over the $N_{\rm rep}=500$ Monte Carlo replicas.
%
We observe that the fit parameters for both $b$ and $E_{\rm BG}$ are stable with respect
to variations of $\Delta E_I$, with any shift in the central value contained within the
uncertainty bands.
%
We can therefore conclude that our approach is robust with respect to the choice of its
hyper-parameters.


%%%%%%%%%%%%%%%%%%%%%%%%%%%%%%%%%%%%%%%%%%%%%%%%%%%%%%%%%%
\begin{figure}[H]
\begin{centering}
  \includegraphics[width=0.99\linewidth]{plots/Stability_plots_sp14_smooth.pdf} 
  \caption{\small The values of the exponent $b$ (left)
    and the bandgap energy $E_{\rm BG}$ (right panel) from the model Eq.~(\ref{eq:I1})
    obtained from the subtracted spectrum sp14 as $\Delta E_{\rm I}$ is varied by $\pm 0.2$ eV
    around its optimal value, indicated by the horizontal dashed line.
  }
\label{fig:bvalues_sampleA}
\end{centering}
\end{figure}
%%%%%%%%%%%%%%%%%%%%%%%%%%%%%%%%%%%%%%%%%%%%%%%%%%%%%%%%%%%%%


The final values for $E_{\rm BG}$ and $b$ obtained in the analysis for spectrum 4 and 5 are
\bea
E_{\rm BG}^{(4)} = 1.6_{-0.2}^{+0.3}\,{\rm eV} \, ,\quad b^{(4)} = 1.3_{-0.7}^{+0.3} (\#4)  \, ,\\
E_{\rm BG}^{(5)}  = 1.6_{-0.2}^{+0.2}\,{\rm eV} \, ,\quad b^{(5)} = 1.3_{-0.5}^{+0.3} (\#5) \, .
\eea
We thus find that for this specific region of the WS$_2$ nanoflowers
the model fit to the subtracted EEL spectrum exhibits a clear preference
for an indirect bandgap (where $b\simeq 1.5$), though a direct one ($b\simeq 0.5$)
cannot be excluded within uncertainties.
%
This result is consistent with the theoretical expectations of the local
electronic properties of bulk WS$_2$.
%
Further, the value of $E_{\rm BG}$ is in agreement with previous determinations
in the same material at the bulk level, such as those collected in Table~\ref{table:bgvalues}.
%
Consistent results are obtained for other locations over the specimen
where spectra have been recorded. 
%
To demonstrate this, we show in Fig.~\ref{fig:bgstability} the fitted values 
for $E_{\rm BG}$ and $b$ for all spectra 
in Sample A, all evaluated using $\Delta E_I$=1.8 eV for the model training.
%
The error of the bandgap fit is different between the spectra, however
what is most important is the fact that central values are stable within 
uncertainty bands.
%
This implies that we find a favor for an indirect bandgap for each of
the locations on Sample A, which is what we would expect from 
theoretical expectations of bulk WS$_2$.

To the best of our knowledge,
these results represent the first EELS bandgap analysis of WS$_2$ nanostructures
whose crystalline structure is based on mixed 2H/3R polytypes.

%%%%%%%%%%%%%%%%%%%%%%%%%%%%%%%%%%%%%%%%%%%%%%%%%%%%%%%%%%
\begin{figure}[H]
\begin{centering}
  \includegraphics[width=0.9\linewidth]{plots/bg_stability.pdf} 
  \caption{The values of the exponent $b$ (left)
    and the bandgap energy $E_{\rm BG}$ (right panel) from the model Eq.~(\ref{eq:I1})
    obtained from all the subtracted spectra in Sample A, 
    for $\Delta E_{\rm I}$ at its optimal value (1.8 eV). 
  }
\label{fig:bgstability}
\end{centering}
\end{figure}
%%%%%%%%%%%%%%%%%%%%%%%%%%%%%%%%%%%%%%%%%%%%%%%%%%%%%%%%%%%%%

One final remark to make is that the polynomial fit to the bandgap onset
comes with the introduction of new parameters to be tuned:
one has to define the fitting range and bounds of the fit, 
introducing some degree of arbitrariness to the procedure.
%
The influence of the chosen fit regime to the polynomial fit
can be observed in Fig~\ref{fig:fitregime} below.
%
Here, we have performed the polynomial fit using three different
upper values of the energy loss: $\Delta E_I$ + 0.5, 0.7, and 0.9 eV. 
%
The results for the exponents $b$ are 1.37, 1.76 and 1.63 respectively. 
%
These values show a clear preference for an indirect bandgap, 
irrespective of the fitting regime. 
%
Values for E$_{\rm bg}$ are stable and equal to 1.6 eV for all cases.

%%%%%%%%%%%%%%%%%%%%%%%%%%%%%%%%%%%%%%%%%%%%%%%%%%%%%%%%%%
\begin{figure}[H]
\begin{centering}
  \includegraphics[width=0.7\linewidth]{plots/bandgap_different_regimes.pdf} 
  \caption{Bandgap fits for different fitting regimes: 
  $\Delta E_{\rm fit}= [0, \Delta E_{\rm lim,up}]$ 
  with $\Delta E_{\rm lim,up}$ = $\Delta E_I$ + 0.5 (green), 0.7 (red), and 0.9 (blue). 
  We find for the exponent $b$ = 1.37 (green), 1.76 (red), and 1.63 (blue).
  The bandgap energy is 1.6 eV irrespective of the fitting regime.
  }
\label{fig:fitregime}
\end{centering}
\end{figure}
%%%%%%%%%%%%%%%%%%%%%%%%%%%%%%%%%%%%%%%%%%%%%%%%%%%%%%%%%%%%%
%

It is important to note that up to the subtraction procedure, 
methods are almost completely free of manual finetuning
of parameters.
%
It is only for the analysis of the bandgap onset that 
few new tuning parameters are introduced, necessary to perform 
the bandgap onset fit, and changing these parameters
yields to slightly different fit results.
%
This is a general flaw of the bandgap fitting procedure 
which is separate from our subtraction method.


\subsection{Mapping excitonic transitions in sample B}

We now discuss the results of the bandgap analysis of sample B,
taking location \#4 in Fig.~\ref{fig:ws2positions} as representative spectrum; 
again, compatible results are found for the rest of positions.
%
For the application of our ZLP subtraction strategy to the EEL spectra recorded in specimen B
of the WS$_2$ nanoflowers (bottom panels
in Fig.~\ref{fig:ws2positions}), the same criterion
based on the derivative ratio Eq.~(\ref{eq:rder}) to select the hyper-parameter $\Delta E_{\rm I}$ was
used.
%
In this case, one finds a value of $\Delta E_{\rm I}\simeq 1.45$ eV,
which is somewhat lower than the corresponding value obtained for sample A.
%
The left panel of Fig.~\ref{fig:SubtractedEELS_sampleB} displays
the original and subtracted spectra corresponding to the representative
location \#4 of sample B together with the predictions of the ZLP model.
%
%%%%%%%%%%%%%%%%%%%%%%%%%%%%%%%%%%%%%%%%%%%%%%%%%%%%%%%%%%%%%%%%%%%%%%%
\begin{figure}[H]
\begin{centering}
  \includegraphics[width=0.98\linewidth]{plots/subtractedEELS_plot_sampleB_sp4.pdf}
  \caption{Left: the original
     and subtracted EEL spectra corresponding to location \#4 of sample B in Fig.~\ref{fig:ws2positions},
     together with the predictions of the ZLP model.
     %
     The bands indicate the 68\% confidence level uncertainties.
     %
     Right: comparison of the ZLP-subtracted spectra from locations \#4, \#5, and \#6 in sample B
     together with the corresponding model uncertainties.
    %
     Note how several features of the subtracted spectra, in particular
     the peaks at $\Delta E\simeq 1.5$,
    1.7 and 2.0 are eV, are common across the three locations.
  }
\label{fig:SubtractedEELS_sampleB}
\end{centering}
\end{figure}
%%%%%%%%%%%%%%%%%%%%%%%%%%%%%%%%%%%%%%%%%%%%%%%%%%%%%%%%%%%%%%%%%%%%%%%%%%

The main difference with respect to the spectra recorded in sample A is that in sample B,
well-defined features (peaks) appear in the subtracted spectrum already for
very small values of $\Delta E$.
%
In particular, we observe two pronounced peaks at $\Delta E\simeq 1.5$ and 2.0 eV and a
softer one near $\Delta E \simeq 1.7$ eV.
%
Additional features arise also for higher energy losses.
%
There are two main explanations for the observed differences between the spectra recorded
in sample A and B.
%
The first one is that, while sample A is much thicker (bulk), sample B corresponds
to thin, overlapping petals whose thicknesses can be as small as a few monolayers.
%
The second is that the EELS measurements taken in sample A used a TEM without monochromator,
while those in sample B were recorded with a monochromator, thereby achieving a superior
spectral resolution.
%
The average FWHM on Sample B is 87 meV, to be compared with 470 meV of sample A (see
Table~\ref{table:sampledata}).
%
This difference in morphology between the specimens, together
with the operation conditions of the TEM, could account for 
the observed differences between the two sets of spectra.

It is worth noting that our ZLP parametrisation and subtraction strategy performs well
for all the spectra under consideration, irrespective of the spectral resolution of the TEM used
for their acquisition.
%
By comparing the left panel of Fig.~\ref{fig:SubtractedEELS_sampleB} with Fig.~\ref{fig:sp14_subtracted_spectrum}, 
one observes that  model uncertainties are larger in the latter case (sample A) 
than in the former (sample B), as expected from the
superior spectral resolution of the EELS measurements taken on sample B.
%
Nevertheless, the same approach has been used in both cases without the need of any fine-tuning
or adjustments: of course, if the input
spectra have been recorded with higher spectral resolution, the resulting ZLP model uncertainties
will improve accordingly without the need for changing the procedure itself.\\

Given that the
well-defined spectral features present in Fig.~\ref{fig:SubtractedEELS_sampleB}
appear close to the onset of the inelastic emissions, $I_{\rm inel}(\Delta E)$,
these spectra are not suitable for bandgap determination analyses.
%
The reason is that the method of~\cite{Rafferty:2000}
used in sample A is only applicable under the assumption that there is a sufficiently wide region in $\Delta E$
after the onset of $I_{\rm inel}$ to perform the polynomial fit of Eq.~(\ref{eq:I1}).
%
This is clearly not possible  for the spectra recorded in sample B, and indeed model fits restricted to $\Delta E\le 1.4$ eV
are numerically very unstable.
%
Instead of studying the bandgap properties, it is interesting to exploit the ZLP-subtracted results of sample B
to characterise the local
excitonic transitions of polytypic 2H/3R WS$_2$
that are known to arise in the ultra-low-loss region of the spectra.

Before being able to do this, however, one has to deal with the possible objection
that the peaks present in the left panel of Fig.~\ref{fig:SubtractedEELS_sampleB} are not
genuine features, but rather fluctuations due to insufficient statistics
that should be smoothed out before this region can be analysed.
%
To tackle this concern, the right panel of Fig.~\ref{fig:SubtractedEELS_sampleB}
displays a comparison of the ZLP-subtracted spectra recorded in the 
(spatially separated) locations \#4, \#5 and \#6
in sample B together with their model uncertainties.
%
Both the position and the widths of the peaks at $\Delta E\simeq 1.5$,
1.7 and 2.0 eV remain stable, confirming that these
are genuine physical features rather than fluctuations.
%
In order to even further establish the presence of these peaks, 
we have created the right panel of Fig.~\ref{fig:SubtractedEELS_sampleB}
but now for a higher value of $\Delta E_{\rm I}$. 
%
Results are shown in Fig.~\ref{fig:subtracted_spectra_comp}, now for two values of the 
hyperparameter $\Delta E_{\rm I}$, 1.45 eV (left) and 1.55 eV (right panel).
%
Note how these low-loss features of the subtracted spectra 
are still common across the three spectra, demonstrating that these
peaks are constantly present
also when using different hyper-parameters.

%%%%%%%%%%%%%%%%%%%%%%%%%%%%%%%%%%%%%%%%%%%%%%%%%%%%%%%%%%%%%%%%%%%%%%%
\begin{figure}[H]
\begin{centering}
  \includegraphics[width=0.99\linewidth]{plots/subtracted_spectra_comp.pdf}
  \caption{The ZLP-subtracted spectra from sample B corresponding to locations \#4, \#5, and \#6
    from Fig.~\ref{fig:ws2positions} together with the corresponding model uncertainties.
    %
    Results are shown for two values of the hyperparameter $\Delta E_{\rm I}$,
    1.45 eV (left) and 1.55 eV (right panel).
    %
    Note how features of the subtracted spectra such as the peaks as $\Delta E\simeq 1.5$,
    1.7 and 2.0 are common across the three spectra.
  }
\label{fig:subtracted_spectra_comp}
\end{centering}
\end{figure}
%%%%%%%%%%%%%%%%%%%%%%%%%%%%%%%%%%%%%%%%%%%%%%%%%%%%%%%%%%%%%%%%%%%%%%%%%%

These peaks in the ultra-low-loss region of the ZLP-subtracted EELS spectra recorded on thin, polytypic
WS$_2$ nanostructures can be traced back to excitonic transitions.
%
Their origin can be attributed to the formation of an electron-hole pair mitigated
by the dielectric screening from the surrounding lattice~\cite{Hanbicki:2016}.
%
In nanostructures with reduced dimensionality as well as in single layers of TMDs, 
exciton peaks arise with binding energies
up to ten times larger than for bulk structures.
%
In the optical spectra of TMDs, two strongly pronounced resonances denoted by A and B
excitons are often observed, appearing at binding energies of 300 and
500 meV below the true band gap of the material~\cite{Karivaj:2019}.
%
Interestingly, this is in agreement with the features observed in Fig.~\ref{fig:subtracted_spectra_comp} at 
$\Delta E\simeq 1.5$ and 1.7 eV, which is exactly 300-500 meV below the true bandgap value
expected for 2D structures of WS$_2$, see Table~\ref{table:bgvalues}.  

We  conclude that ZLP-subtracted spectra in sample B allow for
a clean mapping of the exciton peaks present in the WS$_2$ nanoflowers
down to $\Delta E\simeq 1.5$ eV together with
the associated uncertainty estimate.
%
Further insights concerning the relationship between the exciton peaks in the ultra-low-loss region
and the underlying crystalline structure and specimen morphology could be obtained
by combining our findings with {\it ab initio} calculations such as those based on
density functional theory.


\newpage

% Summary and outlook
%%%%%%%%%%%%%%%%%%%%%%%%%%%%%%%%%%%%%%%%%
\section{Summary and outlook}
%%%%%%%%%%%%%%%%%%%%%%%%%%%%%%%%%%%%%%%%%
\label{sec:summary}

In this work we have presented a novel, model-independent strategy to parametrise and subtract
the omnipresent zero-loss peak that dominates the low-loss region
of EEL spectra.
%
Our strategy is based on machine learning techniques and provides a faithful estimate of the
uncertainties associated to both the input data and the procedure itself,
which can  then be propagated to physical predictions without any assumptions or approximations.
%
We have demonstrated how, in the case of vacuum spectra, our approach
is sufficiently flexible to accomodate several input variables corresponding
to different operating conditions of the microscope, such as the exposure time and beam energy.
%
Further, we are able to reliably interpolate and
extrapolate our predictions, {\it e.g.} for the expected FWHM of the ZLP,
to operating conditions not included in the training dataset.
%
When applied to spectra recorded on specimens, our approach
makes it possible to robustly disentangle the ZLP contribution from
those arising from inelastic interactions with the sample.
%
Thanks to this subtraction procedure, one can fully exploit
the valuable physical information contained in the ultra low-loss region of
the spectra. \\

As a proof of concept, we have applied the ZLP subtraction
strategy to EEL spectra recorded in two samples of WS$_2$ nanoflowers characterised by a
2H/3R polytypic crystalline structure.
%
Measurements taken in the first sample, representing a relatively thick region of WS$_2$ (bulk material),
were used to determine
the local value of the bandgap energy $E_{\rm BG}$
and to assess whether the nature of this bandgap is direct or indirect.
%
A model fit to the onset of the inelastic intensity distribution leads to
a bandgap energy
$E_{\rm BG} \simeq 1.6^{+0.3}_{-0.2}\,{\rm eV}$ and 
exhibits a clear preference for an indirect bandgap.
%
Our findings are consistent with previous studies, both of theoretical
and of experimental nature, concerning the bandgap structure of bulk WS$_2$.

Subsequently, we have applied our method to a thinner sample of the WS$_2$ nanoflowers,
specifically a region composed by overlapping petals with varying
thicknesses that can be as small as a few monolayers.
%
We have demonstrated how for such specimens one can exploit the ZLP-subtracted results
to characterise the local excitonic transitions that arise in the ultra-low-loss region.
%
By charting the bandgap region region of 2H/3R polytypic WS$_2$,
we identify two strong peaks at $\Delta E\simeq 1.5$ and 2 eV
(and a softer one at 1.7 eV) and we show how
these features are consistent when comparing
spatially-separated locations in sample B, 
independent of our choice of hyper-parameters.

The power of this method is that it provides an associated uncertainty estimate,
which makes it possibile to robustly establish the statistical significance
of each of these features in the ultra-low-loss region.\\

The approach presented in this work could be extended
in several directions.
%
First of all, it would be interesting to test its robustness when additional
operating conditions of the microscope are included as input variables,
{\it e.g.} arperture width or temperature,
and to verify to which extent the ZLP parametrisations obtained for an specific microscope
can be generalised to a completely different TEM.
%
Further, a non-trivial cross-check of our method would be to validate
our predictions for other operating conditions of the microscope with actual measurements, such
as the FWHM as a function of the beam energy $E_b$ or the exposure time
$t_{\rm exp}$ as shown in Fig.~\ref{fig:extrapolbeam}.

Concerning the physical interpretation of the low-loss region of EEL
spectra, our method could be applied to study the bandgap properties 
for different types
of nanostructures built upon TMD materials, such as MoS$_2$ nanowalls~\cite{nanowalls}
and vertically-oriented nano-sheets~\cite{Bolhuis:2020}.
%
It would also be interesting to use a sample which is known to exhibit
a direct bandgap without dominating excitonic transitions in the low-loss regime, 
to verify that the bandgap fitting procedure works 
also for this case.
%
In addition to bandgap characterisation, this ZLP-subtraction
strategy should allow the detailed study
of other phenomena relevant for the interpretation of the low-loss
region such as plasmons, excitons, phonon interactions, and
intra-band transitions.
%
Once could also further exploit the observed peaks in the ultra-low-loss
regime in WS$_2$ monolayers, infer their binding energies
and verify if these peaks indeed correspond to the expected
exciton transitions.
%
Furthermore, the subtracted EEL spectra can be used to further characterise
local electronic properties by means of the
evaluation of the dielectric function and its associated
uncertainties in terms of the Kramers-Kronig relations.

Another possible application of the strategy presented in this work would be the automation of
the study of spectral TEM images,
such as those displayed in the right panels of Fig.~\ref{fig:ws2positions},
where each pixel contains an individual EEL spectrum.
%
Here machine learning methods would provide a useful automated method
to identify relevant features of the spectra with minimal
human intervention, since there is no need to process each spectrum individually.
%
It can then be used to map
the evolution of these features as we move along different regions of the
nanostructure.
%
Such an approach would combine two important families of machine learning algorithms: 
regression, in order to quantify the properties of spectral
features such as width and significance, and classification, to identify categories
of features across the spectral image.

\newpage
\section*{Methods}

{The EEL spectra used for the training of the vacuum ZLP model presented in Sect.~\ref{sec:results_vacuum} 
were collected in a ARM200F Mono-JEOL microscope equipped with a GIF continuum spectrometer and operated at 
60 kV and 200 kV. 
%
For these measurements, a slit in the monochromator of 2.8 $\mu$m was used.
%
The TEM and EELS measurements acquired in Specimen A for the results presented in
Sect.~\ref{sec:results_sample} were recorded in a JEOL 2100F microscope with a cold field-emission
gun equipped with aberration corrector operated at 60 kV. A Gatan GIF Quantum was used for
the EELS analyses. The convergence and collection semi-angles were 30.0 mrad and 66.7 mrad respectively.
%
The TEM and EELS measurements acquired for Specimen B in Sect.~\ref{sec:results_sample}
were recorded using a JEM ARM200F monochromated microscope operated at 60 kV and equipped with
a GIF quantum ERS. The convergence and collection semi-angles were 24.6 mrad and 58.4 mrad respectively
in this case, and the aperture of the spectrometer was set to 5 mm.}




\appendix
\input{app-code.tex}
\newpage

\bibliography{thesis}
\newpage

\noindent {\bf \large Acknowledgements}\\[1.0cm]

\noindent My endless gratitude goes to my official and unofficial supervisors
Sonia and Juan.
%
Sonia, your incredible enthusiasm, support and optimism
have been a driving force throughout my project.
%
Unfortunately, due to Covid, it only lasted one month that
we could spend our coffee breaks 
in real life together with Luigi. 
%
I'm glad that we shared many more virtually.\\

\noindent Thank you also for introducing a mysterious professor from 
Amsterdam to my project, who would later become 
my most-dialed contact in my contact list.
%
Juan, I'm so thankful for your engagement and 
never-ending pool of ideas, for all the time that you put
in our collaboration, and also for being critical and
down to earth. 
%
I think we made a super good team.\\

\noindent Luigi, thank you for regularly checking up on me, sending me
random memes and introducing me in the PhD bar when it was 
still allowed, although still I wasn't allowed there.\\

\noindent Finally, I want to thank my parents, brother and friends 
for their support along the way, always there to offer some love and distraction
when my eyes were squared. \\

Laurien





\vspace{0.6cm}


\end{document}
