%%%%%%%%%%%%%%%%%%%%%%%%%%%%%%%%%%%%%%%%%
\section{Summary and outlook}
%%%%%%%%%%%%%%%%%%%%%%%%%%%%%%%%%%%%%%%%%
\label{sec:summary}

In this work we have presented a novel, model-independent strategy to parametrise and subtract
the omnipresent zero-loss peak that dominates the low-loss region
of EEL spectra.
%
Our strategy is based on machine learning techniques and provides a faithful estimate of the
uncertainties associated to both the input data and the procedure itself,
which can  then be propagated to physical predictions without any assumptions or approximations.
%
We have demonstrated how, in the case of vacuum spectra, our approach
is sufficiently flexible to accomodate several input variables corresponding
to different operating conditions of the microscope, such as the exposure time and beam energy.
%
Further, we are able to reliably interpolate and
extrapolate our predictions, {\it e.g.} for the expected FWHM of the ZLP,
to operating conditions not included in the training dataset.
%
When applied to spectra recorded on specimens, our approach
makes it possible to robustly disentangle the ZLP contribution from
those arising from inelastic interactions with the sample.
%
Thanks to this subtraction procedure, one can fully exploit
the valuable physical information contained in the ultra low-loss region of
the spectra. \\

As a proof of concept, we have applied the ZLP subtraction
strategy to EEL spectra recorded in two samples of WS$_2$ nanoflowers characterised by a
2H/3R polytypic crystalline structure.
%
Measurements taken in the first sample, representing a relatively thick region of WS$_2$ (bulk material),
were used to determine
the local value of the bandgap energy $E_{\rm BG}$
and to assess whether the nature of this bandgap is direct or indirect.
%
A model fit to the onset of the inelastic intensity distribution leads to
a bandgap energy
$E_{\rm BG} \simeq 1.6^{+0.3}_{-0.2}\,{\rm eV}$ and 
exhibits a clear preference for an indirect bandgap.
%
Our findings are consistent with previous studies, both of theoretical
and of experimental nature, concerning the bandgap structure of bulk WS$_2$.

Subsequently, we have applied our method to a thinner sample of the WS$_2$ nanoflowers,
specifically a region composed by overlapping petals with varying
thicknesses that can be as small as a few monolayers.
%
We have demonstrated how for such specimens one can exploit the ZLP-subtracted results
to characterise the local excitonic transitions that arise in the ultra-low-loss region.
%
By charting the bandgap region region of 2H/3R polytypic WS$_2$,
we identify two strong peaks at $\Delta E\simeq 1.5$ and 2 eV
(and a softer one at 1.7 eV) and we show how
these features are consistent when comparing
spatially-separated locations in sample B, 
independent of our choice of hyper-parameters.

The power of this method is that it provides an associated uncertainty estimate,
which makes it possibile to robustly establish the statistical significance
of each of these features in the ultra-low-loss region.\\

The approach presented in this work could be extended
in several directions.
%
First of all, it would be interesting to test its robustness when additional
operating conditions of the microscope are included as input variables,
{\it e.g.} arperture width or temperature,
and to verify to which extent the ZLP parametrisations obtained for an specific microscope
can be generalised to a completely different TEM.
%
Further, a non-trivial cross-check of our method would be to validate
our predictions for other operating conditions of the microscope with actual measurements, such
as the FWHM as a function of the beam energy $E_b$ or the exposure time
$t_{\rm exp}$ as shown in Fig.~\ref{fig:extrapolbeam}.

Concerning the physical interpretation of the low-loss region of EEL
spectra, our method could be applied to study the bandgap properties 
for different types
of nanostructures built upon TMD materials, such as MoS$_2$ nanowalls~\cite{nanowalls}
and vertically-oriented nano-sheets~\cite{Bolhuis:2020}.
%
It would also be interesting to use a sample which is known to exhibit
a direct bandgap without dominating excitonic transitions in the low-loss regime, 
to verify that the bandgap fitting procedure works 
also for this case.
%
In addition to bandgap characterisation, this ZLP-subtraction
strategy should allow the detailed study
of other phenomena relevant for the interpretation of the low-loss
region such as plasmons, excitons, phonon interactions, and
intra-band transitions.
%
Once could also further exploit the observed peaks in the ultra-low-loss
regime in WS$_2$ monolayers, infer their binding energies
and verify if these peaks indeed correspond to the expected
exciton transitions.
%
Furthermore, the subtracted EEL spectra can be used to further characterise
local electronic properties by means of the
evaluation of the dielectric function and its associated
uncertainties in terms of the Kramers-Kronig relations.

Another possible application of the strategy presented in this work would be the automation of
the study of spectral TEM images,
such as those displayed in the right panels of Fig.~\ref{fig:ws2positions},
where each pixel contains an individual EEL spectrum.
%
Here machine learning methods would provide a useful automated method
to identify relevant features of the spectra with minimal
human intervention, since there is no need to process each spectrum individually.
%
It can then be used to map
the evolution of these features as we move along different regions of the
nanostructure.
%
Such an approach would combine two important families of machine learning algorithms: 
regression, in order to quantify the properties of spectral
features such as width and significance, and classification, to identify categories
of features across the spectral image.

\newpage
\section*{Methods}

{The EEL spectra used for the training of the vacuum ZLP model presented in Sect.~\ref{sec:results_vacuum} 
were collected in a ARM200F Mono-JEOL microscope equipped with a GIF continuum spectrometer and operated at 
60 kV and 200 kV. 
%
For these measurements, a slit in the monochromator of 2.8 $\mu$m was used.
%
The TEM and EELS measurements acquired in Specimen A for the results presented in
Sect.~\ref{sec:results_sample} were recorded in a JEOL 2100F microscope with a cold field-emission
gun equipped with aberration corrector operated at 60 kV. A Gatan GIF Quantum was used for
the EELS analyses. The convergence and collection semi-angles were 30.0 mrad and 66.7 mrad respectively.
%
The TEM and EELS measurements acquired for Specimen B in Sect.~\ref{sec:results_sample}
were recorded using a JEM ARM200F monochromated microscope operated at 60 kV and equipped with
a GIF quantum ERS. The convergence and collection semi-angles were 24.6 mrad and 58.4 mrad respectively
in this case, and the aperture of the spectrometer was set to 5 mm.}


