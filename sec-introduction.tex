\section{Introduction}
\label{sec:introduction}

The appreciation for the family of two-dimensional (2D) materials has grown rapidly since the 
first isolation of graphene~\cite{Novoselov:2004}.
%
The properties of these two-dimensional materials are usually very different from their 3D counterparts, 
offering functionalities for novel nanodevices that are not accessible from other heterostructures.
%
The combination of optically active semiconducting layers for light-emitting devices,
the implementation of indirect-to-direct bandgap transition 
materials for flexible electronics, the effect of chemical compositions on many-body instabilities such as
superconductivity~\cite{Novoselov:2016},
%
these are just a few examples of the relevance to exploit the properties of these 2D materials. 
%
However, realizing the full potential of any materials system requires knowing the precise 
electronic, structural and chemical information
at high resolution.

Thanks to recent instrumentation breakthroughs in electron microscopy,
such as electron monochromators~\cite{Terauchi:2005, Freitag:2005} and aberration correctors~\cite{Haider:1998},
it becomes possible to map these properties with unexampled spatial and spectral resolution.
%
Specifically by means of electron energy loss spectroscopy (EELS), it becomes possible to study the local
electronic properties of nanomaterials
down to the atomic scale~\cite{Geiger:1967}, and this way to explore various
important phenomena from the characterisation of bulk and surface plasmons~\cite{Daniels:2003, Schaffer:2008}, 
excitons~\cite{Erni:2005},
phonons~\cite{Ibach:1980}, and inter- and intra-band transitions~\cite{Rafferty:1998},
to the determination of the bandgap energy and the band structure~\cite{Stoger:2008}.

Particulary important information about the material's characteristics can be extracted from studying the
low-loss region of EEL spectra,
defined by electrons that have lost
less than a few eV ($\Delta E\lsim 5$ eV) following their inelastic interactions
with the sample.
%
However, an omnipresent feature called the zero-loss peak (ZLP) dominates
the low-loss region of the spectra. 
%
This narrow, high intensity peak is centered at energy losses
of $\Delta E\simeq 0$, is often asymmetric and its tails extend significantly 
beyond the FWHM. 
%
This peak results from the fact that 
the majority of the incident electron beam will traverse the electron-penetrable sample
either without interactions or scattering only elastically with the 
sample's crystalline lattice, therefore losing little to no energy
and ending up in the ZLP.
%
Since in the low-loss region, the contribution from the ZLP tail
overwhelms those from the inelastic interactions between
the incident electrons and the specimen,
relevant signals of low-loss phenomena such as excitons,
phonons, and intra-band transitions risk being drowned
in the ZLP tail~\cite{Abajo:2010}.
%
An accurate removal of the ZLP
contribution is thus crucial in order to efficiently chart and identify the features
of the low-loss region in EEL spectra. 


Several approaches to ZLP subtraction have been put forward in 
literature~\cite{Rafferty:2000, Erni:2005, Stoger:2008, Egerton:1996,Dorneich:1998, Benthem:2001, Lazar:2003}.
%
These are often based on specific model assumptions about the ZLP properties, specifically
concerning its parametric functional dependence on the electron energy loss $\Delta E$,
from Lorentzian~\cite{Dorneich:1998}
and power laws~\cite{Erni:2005} to more general multiple-parameter functions~\cite{Benthem:2001}.
%
Another approach is based on mirroring the $\Delta E <0$ region of the spectra, assuming
that the $\Delta E>0$ region is fully symmetric~\cite{Lazar:2003}.
%
More recent studies use integrated software applications for background subtraction 
methods~\cite{Egerton:10.1016/S0304-3991(01)00155-3, Held:2020, Granerod:2018, Fung:2020}.
%
These subtraction methods are however affected by three main limitations.
%
Firstly, they rely on specific model assumptions {\it e.g.} with
the choice of a specific fit function, introducing a methodological
bias whose size is difficult to quantify.
%
Secondly, they lack an estimate of the associated uncertainties, which in turn affects
the reliability of any physical interpretations of the low-loss region such as
band gap extraction.
%
Thirdly, methodological choices such as fitting ranges introduce a significant degree of
arbitrariness to the procedure.\\

In this work we bypass these limitations by developing a model-independent strategy
that allows for a multidimensional determination of the ZLP
with a faithful uncertainty estimate.
%
Our approach is based on machine learning (ML) techniques
developed in high-energy physics to study
quark and gluon substructure of protons
particle collisions~\cite{Ball:2008by,Ball:2012cx,Ball:2014uwa,Ball:2017nwa}.
%
This technique is based on the Monte Carlo replica method to construct a probability
distribution in the space of the experimental data (here the ZLP) and to use artificial
neural networks as unbiased interpolators to parametrise the ZLP.
%
The result is
a prediction of the ZLP intensity based on its input variables,
without the need to make specific model assumptions or approximations,
which can be used to subtract its contribution to EEL spectra while
propagating the associated uncertainties.
%
Furthermore, we can extrapolate this ZLP parametrisation to other TEM
operating conditions beyond those included in the training dataset.\\

Our work is divided into two main parts.
%
In the first, we construct a ML model of ZLP spectra taken
in vacuum that is able to accommodate an arbitrary number of input
variables corresponding to different operation settings of the TEM, 
{\it e.g.} exposure time and beam energy
%
We demonstrate how the model describes successfully the
input spectra and we assess its extrapolation for other operating
conditions that were not used for training.

In the second part, we construct a one-dimensional model
of the ZLP as a function of the energy loss from spectra acquired on
tungsten disulfide (WS$_2$) nanostructures~\cite{SabryaWS2}.
%
The resulting subtracted spectra are used to determine
the value and type of the WS$_2$ bandgap
and its dependence on the underlying crystalline morphology, 
and to demonstrate how one can exploit the ZLP-subtracted results
to characterise features arising in the very-low-loss region.

This work is organized as follows.
%
First of all, in Sects.~\ref{sec:tmd} and~\ref{sec:eels}
we discuss the intriguing class of transition metal 
dichalcogenide (TMD) materials, with emphasis on WS$_2$,
and we review the main features of the EELS technique.
%
Sect.~\ref{sec:nn} contains
the fundamentals of neural networks and the principles
of supervised machine learning.
%
In Sect.~\ref{sec:methodology} we describe our machine learning methodology
for the ZLP parametrisation.
%
Sects.~\ref{sec:results_vacuum} and~\ref{sec:results_sample} contain
our results for the ZLP parametrisation for spectra acquired
in vacuum and in samples respectively, which in the latter
case allows us to study the local band structure properties
of the WS$_2$ nanoflowers.
%
Finally in Sect.~\ref{sec:summary} we summarise
and outline possible future developments.
%
Our results have been obtained with an open-source {\sc Python} code,
{\tt EELfitter}, presented in App.~\ref{sec:installation}
together with some installation and usage instructions.
